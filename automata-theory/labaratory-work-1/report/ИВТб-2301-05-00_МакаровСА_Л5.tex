\documentclass[a4paper,14pt]{extarticle}

\usepackage[a4paper,top=20mm,bottom=20mm,left=30mm,right=10mm]{geometry}
\usepackage[T1,T2A]{fontenc}
\usepackage[utf8]{inputenc}
\usepackage[russian]{babel}
\usepackage{indentfirst}
\usepackage{titlesec}
\usepackage{graphicx}
\usepackage{verbatim}
\usepackage{fancyvrb}

\renewcommand{\baselinestretch}{1.3}
\titleformat{\section}{\normalsize\bfseries}{\thesection}{1em}{}
\titleformat{\subsection}{\normalsize\bfseries}{\thesection}{1em}{}
\setlength{\parindent}{12.5mm}

\begin{document}

  \newpage\thispagestyle{empty}
  \begin{center}
    \MakeUppercase{
      Министерство науки и высшего образования Российской Федерации\\
      Федеральное государственное бюджетное образовательное учреждение высшего образования\\
      <<Вятский Государственный Университет>>\\
    }
    Институт математики и информационных систем\\
    Факультет автоматики и вычислительной техники\\
    Кафедра электронных вычислительных машин
  \end{center}
  \vfill

  \begin{center}
    Отчет по лабораторной работе №1\\
    по дисциплине\\
    <<Теория автоматов>>\\
  \end{center}
  \vfill

  \noindent
  \begin{tabular}{ll}
    Выполнил студент гр. ИВТб-2301-05-00 \hspace{5mm} &
    \rule[-1mm]{25mm}{0.10mm}\,/Макаров С.А./\\
    
    Преподаватель & \rule[-1mm]{25mm}{0.10mm}\,/Мельцов В.Ю./\\
  \end{tabular}

  \vfill
  \begin{center}
    Киров 2025
  \end{center}

  \newpage
  \section*{Цель}
  Цель лабораторной работы: Получить базовые навыки реализации автомата с алгоритмами <<случайного выбора>>.

  \section*{Задание}
  Имеется 10 рыбаков. У каждого есть садок, в котором от 0 до 9 рыб. Генератор случайно выбирает рыбака, у которого от 0 до 9 рыб, и добавляет 1 рыбу. При срабатывании аварийной лампы выбирается рыбак, у которого от 1 до 9 рыб, и убирается 1 рыба. Если в садке у рыбака 10 рыб, то рыбак перестает ловить рыбу. Программа завершается когда у всех рыбаков полный садок.

  После загрузки программы на экране выводится поле игры и основные кнопки: <<Старт>>, <<Пауза>>, <<Выход>>, также вверху экрана находится меню.

  Пункты меню:
  \begin{itemize}
    \item[--] Файл 
      \begin{itemize}
        \item[--] Открыть -- открывает файл, содержащий настройки
        \item[--] Сохранить -- сохранят настройки в файл
        \item[--] Выход -- выход из программы
      \end{itemize}
    \item[--] Настройки
      \begin{itemize}
        \item[--] Выбор цвета -- открывает окно настройки цвета рыбаков
        \item[--] Начальное заполнение -- открывает окно настройки начального заполнения садков
      \end{itemize}
    \item[--] Справка
      \begin{itemize}
        \item[--] Об авторе -- открывает окно, содержащее сведения об авторе
        \item[--] О программе -- открывает окно, содержащее сведения о программе
      \end{itemize}
  \end{itemize}

  \pagebreak
  \section*{Решение}

  \pagebreak
  \begin{figure}[h]
    \centering
    \includegraphics[width=0.95\linewidth]{img/f-1}
  \end{figure}
  \begin{center}
    Рисунок 2 -- Программа до начала работы
  \end{center}

  \begin{figure}[h]
    \centering
    \includegraphics[width=0.95\linewidth]{img/f-2}
  \end{figure}
  \begin{center}
    Рисунок 3 -- Окно настроек цвета
  \end{center}

  \pagebreak
  \begin{figure}[h]
    \centering
    \includegraphics[width=0.95\linewidth]{img/f-3}
  \end{figure}
  \begin{center}
    Рисунок 4 -- Окно настроек начального заполнения
  \end{center}

  \begin{figure}[h]
    \centering
    \includegraphics[width=0.95\linewidth]{img/f-4}
  \end{figure}
  \begin{center}
    Рисунок 5 -- Окно сведений об авторе
  \end{center}

  \pagebreak
  \begin{figure}[h]
    \centering
    \includegraphics[width=0.95\linewidth]{img/f-4}
  \end{figure}
  \begin{center}
    Рисунок 6 -- Окно сведений о программе
  \end{center}

  \section*{Вывод}
  В ходе выполнения лабораторной работы получены базовые навыки реализации автомата с алгоритмами <<случайно выбора>>. Разработана программа, выполняющая алгоритм работы автомата. Также закреплены знания по реализации пользовательского интерфейса и составлению схем алгоритмов.

\end{document}