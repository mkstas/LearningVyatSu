\documentclass[a4paper,14pt]{extarticle}

\usepackage[a4paper,top=20mm,bottom=20mm,left=30mm,right=10mm]{geometry}
\usepackage[T1,T2A]{fontenc}
\usepackage[utf8]{inputenc}
\usepackage[russian]{babel}
\usepackage{indentfirst}
\usepackage{titlesec}
\usepackage{graphicx}
\usepackage{verbatim}
\usepackage{fancyvrb}

\renewcommand{\baselinestretch}{1.3}
\titleformat{\section}{\normalsize\bfseries}{\thesection}{1em}{}
\titleformat{\subsection}{\normalsize\bfseries}{\thesection}{1em}{}
\setlength{\parindent}{12.5mm}

\begin{document}

  \newpage\thispagestyle{empty}
  \begin{center}
    \MakeUppercase{
      Министерство науки и высшего образования Российской Федерации\\
      Федеральное государственное бюджетное образовательное учреждение высшего образования\\
      <<Вятский Государственный Университет>>\\
    }
    Институт математики и информационных систем\\
    Факультет автоматики и вычислительной техники\\
    Кафедра электронных вычислительных машин
  \end{center}
  \vfill

  \begin{center}
    Отчет по лабораторной работе №2\\
    по дисциплине\\
    <<Теория автоматов>>\\
  \end{center}
  \vfill

  \noindent
  \begin{tabular}{ll}
    Выполнил студент гр. ИВТб-2301-05-00 \hspace{5mm} &
    \rule[-1mm]{25mm}{0.10mm}\,/Макаров С.А./\\
    
    Преподаватель & \rule[-1mm]{25mm}{0.10mm}\,/Мельцов В.Ю./\\
  \end{tabular}

  \vfill
  \begin{center}
    Киров 2025
  \end{center}

  \newpage
  \section*{Цель}
  Получить навыки разработки алгоритма, реализующего автомат, который на основании истории ходов соперников может предсказывать его следющий ход в игре <<Камень, Ножницы, Бумага>>.

  \section*{Задание}
  Разработать алгоритм работы бота для игры <<Камень, Ножницы, Бумага>>, разработать программу, принять участие в турнире.

  \pagebreak
  \section*{Решение}

  \begin{figure}[h]
    \centering
    \includegraphics[width=1\linewidth]{img/schema.png}
  \end{figure}

  \pagebreak
  Исходный код программы, написанный на языке Python представлен ниже:

  \noindent
  \begin{Verbatim}[tabsize=4,fontsize=\small]
ROCK = 1  # Камень
PAPER = 2  # Бумага
SCISSORS = 3  # Ножницы

history = [[], [], []]

def set_parameters(set_count: int, wins_per_set: int) -> None:
    """
    Вызывается один раз перед началом игры.
    Передаёт параметры, с которыми запущен турнир.

    :param set_count: Максимальное количество сетов в игре
    :param wins_per_set: Требуемое количество побед в сете
    """
    pass

def on_game_start() -> None:
    """ Вызывается один раз в начале игры. """
    pass

def beat(move: int) -> int:
    """ Возвращает ход, который бьёт переданный. """
    return {ROCK: PAPER, PAPER: SCISSORS, SCISSORS: ROCK}[move]

def choose(previous_opponent_choice: int) -> int:
    """
    Функция должна вернуть число от 1 до 3, соответствующее фигуре, 
    которую выбрал бот
    (1 - Камень, 2 - Бумага, 3 - Ножницы).

    Передаваемый параметр previous_opponent_choice - число от 1 до 3, выбор 
    противника на предыдущем ходу.
    Самый первый раз за игру, при первом вызове choose, 
    previous_opponent_choice равен 0 (т.к. предыдущих ходов ещё не было).

    :param previous_opponent_choice: Код фигуры, выбранной противником на
    предыдущем ходу
    :return: Код фигуры, которую выбирает бот
    """

    if previous_opponent_choice == 0:
        return PAPER

    history[0].append(previous_opponent_choice)

    if len(history[0]) < 3:
        return PAPER

    pattern = tuple(history[0][-2:])

    for i in range(len(history[0]) - 2):
        if tuple(history[0][i:i+2]) == pattern:
            return beat(history[0][i + 2])

    return PAPER

def on_game_end() -> None:
    """ Вызывается один раз в конце игры. """
    pass
  \end{Verbatim}

  \section*{Вывод}
  В ходе выполнения лабораторной работы разработан алгоритм работы бота для игры <<Камень, Ножницы, Бумага>>, а также разработана программа на языке Python, реализующий данный алгоритм.

\end{document}