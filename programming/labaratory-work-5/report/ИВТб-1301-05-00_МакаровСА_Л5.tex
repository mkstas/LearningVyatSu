\documentclass[a4paper,14pt]{extarticle}

\usepackage[a4paper,top=20mm,bottom=20mm,left=30mm,right=10mm]{geometry}
\usepackage[T1,T2A]{fontenc}
\usepackage[utf8]{inputenc}
\usepackage[russian]{babel}
\usepackage{indentfirst}
\usepackage{titlesec}
\usepackage{graphicx}
\usepackage{verbatim}
\usepackage{fancyvrb}

\renewcommand{\baselinestretch}{1.3}
\titleformat{\section}{\normalsize\bfseries}{\thesection}{1em}{}
\titleformat{\subsection}{\normalsize\bfseries}{\thesection}{1em}{}
\setlength{\parindent}{12.5mm}

\begin{document}

  \newpage\thispagestyle{empty}
  \begin{center}
    \MakeUppercase{
      Министерство науки и высшего образования Российской Федерации\\
      Федеральное государственное бюджетное образовательное учреждение высшего образования\\
      <<Вятский Государственный Университет>>\\
    }
    Институт математики и информационных систем\\
    Факультет автоматики и вычислительной техники\\
    Кафедра электронных вычислительных машин
  \end{center}
  \vfill

  \begin{center}
    Отчет по лабораторной работе №5\\
    по дисциплине\\
    <<Программирование>>\\
  \end{center}
  \vfill

  \noindent
  \begin{tabular}{ll}
    Выполнил студент гр. ИВТб-1301-05-00 \hspace{5mm} &
    \rule[-1mm]{25mm}{0.10mm}\,/Макаров С.А./\\
    
    Руководитель зав. кафедры ЭВМ & \rule[-1mm]{25mm}{0.10mm}\,/Долженкова М.Л./\\
  \end{tabular}

  \vfill
  \begin{center}
    Киров 2025
  \end{center}

  \newpage
  \section*{Цель}
  Цель лабораторной работы: получить базовые сведения о наиболее известных алгоритмах сортировки, изучить принципы работы с текстовыми файлами.

  \section*{Задание}
  \begin{enumerate}
    \item Реализовать сортировку данных с помощью алгоритма подсчётом.
    
    \item Реализовать сортировку данных с помощью поразрядного алгоритма.
    
    \item В обоих случаях необходимо предусмотреть возможность изменения компаратора (реализация компаратора в виде передаваемой в подпрограмму функции).
    
    \item Считывание и вывод данных необходимо производить из текстового файла.
    
    \item Для демонстрации работы программных реализаций самостоятельно подготовить варианты входных данных (при этом объем тестовых файлов должен позволять оценить скорость работы программ).
  \end{enumerate}

  \pagebreak
  \section*{Решение}

  \begin{figure}[h]
    \centering
    \includegraphics[width=0.7\linewidth]{images/s-1-1}
  \end{figure}
  \begin{center}
    Рисунок 1 – Схема алгоритма сортировки подсчетом
  \end{center}

  \pagebreak
  \begin{figure}[h]
    \centering
    \includegraphics[width=0.58\linewidth]{images/s-1-2}
  \end{figure}
  \begin{center}
    Рисунок 2 – Схема алгоритма подпрограммы <<Сортировка>>
  \end{center}

  \pagebreak
  \begin{figure}[h]
    \centering
    \includegraphics[width=0.6\linewidth]{images/s-1-3}
  \end{figure}
  \begin{center}
    Рисунок 3 – Схема алгоритма подпрограммы <<Вверх>>
  \end{center}

  \pagebreak
  \begin{figure}[h]
    \centering
    \includegraphics[width=0.6\linewidth]{images/s-1-4}
  \end{figure}
  \begin{center}
    Рисунок 4 – Схема алгоритма подпрограммы <<Вниз>>
  \end{center}

  \noindent
  \begin{Verbatim}[tabsize=4,fontsize=\small]
#include <stdio.h>
#include <stdlib.h>
void sort(int n, int array[], 
  void(*comparator)(int range, int min, int array[], int count[])) {
  int min = array[0];
  int max = array[0];
  for (int i = 1; i < n; i++) {
    if (array[i] < min) min = array[i];
    if (array[i] > max) max = array[i];
  }
  int range = max - min + 1;
  int* count = (int*)calloc(range, sizeof(int));
  for (int i = 0; i < n; i++) count[array[i] - min]++;
  comparator(range, min, array, count);
}
void comparatorUp(int range, int min, int array[], int count[]) {
  int j = 0;
  for (int i = 0; i < range; i++) {
    while (count[i] > 0) {
      count[i]--;
      array[j++] = i + min;
    }
  }
}
void comparatorDown(int range, int min, int array[], int count[]) {
  int j = 0;
  for (int i = range - 1; i >= 0; i--) {
    while (count[i] > 0) {
      count[i]--;
      array[j++] = i + min;
    }
  }
}
int main() {
  FILE* input = fopen("../input.txt", "r");
  FILE* output = fopen("../output.txt", "w");
  int n = 0;
  fscanf_s(input, "%d", &n);
  int* array = (int*)calloc(n, sizeof(int));
  for (int i = 0; i < n; i++) fscanf_s(input, "%d", &array[i]);
  fclose(input);
  void(*comparator)(int range, int min, int array[], int count[]);
  comparator = comparatorUp;
  sort(n, array, comparator);
  for (int i = 0; i < n; i++) {
    fprintf(output, "%d ", array[i]);
  }
  fclose(output);
  free(array);
  return 0;
}
  \end{Verbatim}

  \pagebreak
  \begin{figure}[h]
    \centering
    \includegraphics[width=0.7\linewidth]{images/s-2-1}
  \end{figure}
  \begin{center}
    Рисунок 5 – Схема алгоритма поразрядной сортировки
  \end{center}

  \pagebreak
  \begin{figure}[h]
    \centering
    \includegraphics[width=0.55\linewidth]{images/s-2-2}
  \end{figure}
  \begin{center}
    Рисунок 6 – Схема алгоритма подпрограммы <<Сортировка>>
  \end{center}

  \pagebreak
  \begin{figure}[h]
    \centering
    \includegraphics[width=0.43\linewidth]{images/s-2-3}
  \end{figure}
  \begin{center}
    Рисунок 7 – Схема алгоритма подпрограммы <<Разряды>>
  \end{center}

  \pagebreak
  \begin{figure}[h]
    \centering
    \includegraphics[width=0.23\linewidth]{images/s-2-4}
  \end{figure}
  \begin{center}
    Рисунок 8 – Продолжение схемы алгоритма подпрограммы <<Разряды>>
  \end{center}

  \pagebreak
  \begin{figure}[h]
    \centering
    \includegraphics[width=0.65\linewidth]{images/s-2-5}
  \end{figure}
  \begin{center}
    Рисунок 9 – Схема алгоритма подпрограммы <<Вверх>>
  \end{center}

  \pagebreak
  \begin{figure}[h]
    \centering
    \includegraphics[width=0.65\linewidth]{images/s-2-6}
  \end{figure}
  \begin{center}
    Рисунок 10 – Схема алгоритма подпрограммы <<Вниз>>
  \end{center}

  \noindent
  \begin{Verbatim}[tabsize=4,fontsize=\small]
#include <stdio.h>
#include <stdlib.h>
void radixSort(int n, int array[]) {
  int max = array[0];
  for (int i = 1; i < n; i++) {
    if (array[i] > max) max = array[i];
  }
  for (int exp = 1; max / exp > 0; exp *= 10) {
    int* temp = (int*)calloc(n, sizeof(int));
    int count[10] = { 0 };
    for (int i = 0; i < n; i++) {
      count[array[i] / exp % 10]++;
    }
    for (int i = 1; i < 10; i++) {
      count[i] += count[i - 1];
    }
    for (int i = n - 1; i >= 0; i--) {
      temp[count[array[i] / exp % 10] - 1] = array[i];
      count[array[i] / exp % 10]--;
    }
    for (int i = 0; i < n; i++) {
      array[i] = temp[i];
    }
  }
}
void sort(int n, int array[], 
  void(*comparator)(int array[], 
    int negCount, int negatives[], int posCount, int positives[])) {
  int* negatives = (int*)calloc(n, sizeof(int));
  int* positives = (int*)calloc(n, sizeof(int));
  int negCount = 0, posCount = 0;
  for (int i = 0; i < n; i++) {
    if (array[i] < 0) {
      negatives[negCount++] = -array[i];
    } else {
      positives[posCount++] = array[i];
    }
  }
  radixSort(negCount, negatives);
  radixSort(posCount, positives);
  for (int i = 0; i < negCount; i++) {
    negatives[i] = -negatives[i];
  }
  comparator(array, negatives, negCount, positives, posCount);
  free(negatives);
  free(positives);
}
void comparatorUp(int array[], int negCount, int negatives[], 
  int posCount, int positives[]) {
  int j = 0;
  for (int i = negCount - 1; i >= 0; i--) {
    array[j++] = negatives[i];
  }
  for (int i = 0; i < posCount; i++) {
    array[j++] = positives[i];
  }
}
void comparatorDown(int array[], int negCount, int negatives[],
  int posCount, int positives[]) {
  int j = 0;
  for (int i = posCount - 1; i >= 0; i--) {
    array[j++] = positives[i];
  }
  for (int i = 0; i < negCount; i++) {
    array[j++] = negatives[i];
  }
}
int main() {
  FILE* input = fopen("../input.txt", "r");
  FILE* output = fopen("../output.txt", "w");
  int n = 0;
  fscanf_s(input, "%d", &n);
  int* array = (int*)calloc(n, sizeof(int));
  for (int i = 0; i < n; i++) fscanf_s(input, "%d", &array[i]);
  fclose(input);
  void(*comparator)(int array[], int negCount, 
    int negatives[], int posCount, int positives[]);
  comparator = comparatorUp;
  sort(n, array, comparator);
  for (int i = 0; i < n; i++) {
    fprintf(output, "%d ", array[i]);
  }
  fclose(output);
  free(array);
  return 0;
}
  \end{Verbatim}

  \section*{Вывод}
  В ходе выполнения лабораторной работы были изучены алгоритмы сортировки подсчётом и поразрядной сортировки по младшим разрядам. Также были изучены принципы работы с текстовыми файлами путём решения предложенных задач.

\end{document}