\documentclass[a4paper,14pt]{extarticle}

\usepackage[a4paper,top=20mm,bottom=20mm,left=30mm,right=10mm]{geometry}
\usepackage[T1,T2A]{fontenc}
\usepackage[utf8]{inputenc}
\usepackage[russian]{babel}
\usepackage{indentfirst}
\usepackage{titlesec}
\usepackage{graphicx}
\usepackage{verbatim}
\usepackage{fancyvrb}

\renewcommand{\baselinestretch}{1.3}
\titleformat{\section}{\normalsize\bfseries}{\thesection}{1em}{}
\titleformat{\subsection}{\normalsize\bfseries}{\thesection}{1em}{}
\setlength{\parindent}{12.5mm}

\begin{document}

  \newpage\thispagestyle{empty}
  \begin{center}
    \MakeUppercase{
      Министерство науки и высшего образования Российской Федерации\\
      Федеральное государственное бюджетное образовательное учреждение высшего образования\\
      <<Вятский Государственный Университет>>\\
    }
    Институт математики и информационных систем\\
    Факультет автоматики и вычислительной техники\\
    Кафедра электронных вычислительных машин
  \end{center}
  \vfill

  \begin{center}
    Отчет по лабораторной работе №4\\
    по дисциплине\\
    <<Программирование>>\\
  \end{center}
  \vfill

  \noindent
  \begin{tabular}{ll}
    Выполнил студент гр. ИВТб-1301-05-00 \hspace{5mm} &
    \rule[-1mm]{25mm}{0.10mm}\,/Макаров С.А./\\
    
    Руководитель зав. кафедры ЭВМ & \rule[-1mm]{25mm}{0.10mm}\,/Долженкова М.Л./\\
  \end{tabular}

  \vfill
  \begin{center}
    Киров 2024
  \end{center}

  \newpage
  \section*{Цель}
  Цель лабораторной работы: освоить навык создания структур данных на статических массивах их использование в решении задач.

  \section*{Задание}
  Даны действительные числа $a_1$, $s_2$, ..., $a_{2n}$ (n >= 2, заранее неизвестно и вводится с клавиатуры). Вычислите: max(min($a_1$, $a_{2n}$), min($a_3$, $a_{2n-2}$), ..., min($a_{2n-1}$, $a_2$)).

  \section*{Решение}
  Для решения данной необходимо использовать двунаправленный список, так как по условию задачи необходимо двигаться в обе стороны списка и среди них искать минимальное число попарно.\\
  \indent К преимуществам двунаправленного списка в сравнении с односвязным списком относятся возможность двигаться в как с начала списка, так и с конца, что позволяет быстрее получать доступ к элементам списка в зависимости от их расположения.\\

  \begin{figure}[h]
    \centering
    \includegraphics[width=0.56\linewidth]{images/s-1}
  \end{figure}
  \begin{center}
    Рисунок 1 – Подпрограмма <<Вставка>>
  \end{center}

  \pagebreak
  \begin{figure}[h]
    \centering
    \includegraphics[width=0.538\linewidth]{images/s-2}
  \end{figure}
  \begin{center}
    Рисунок 2 – Схема алгоритма программы
  \end{center}

  \pagebreak
  \begin{figure}[h]
  \centering
    \includegraphics[width=0.4\linewidth]{images/s-3}
  \end{figure}
  \begin{center}
    Рисунок 3 – Продолжение схемы алгоритма
  \end{center}

  \begin{Verbatim}[tabsize=2]
program solution;

type
  item = record
    data: integer;
    next, prev:byte;
  end;

  list = record
    items:array[1..100] of item;
    head, tail:byte;
  end;
var
  n, k, i, max, min:integer;
  L:list;

procedure push(var L:list; new_el:integer; k:integer);
begin
  L.items[new_el].data := k;
  
  if new_el <> 1 then
  begin
    L.items[new_el].prev := new_el;
    L.items[new_el - 1].next := new_el;
  end;
end;

begin
  L.head := 0;
  L.tail := 0;
  readln(n);
  for i := 1 to 2 * n do
  begin
    read(k);
    push(L, i, k);
  end;
  
  if L.items[1].data > L.items[2 * n].data then
    max := L.items[2 * n].data
  else
    max := L.items[1].data;
  
  for i := 2 to n do
  begin
  if L.items[i + 1].data > L.items[2 * n - (i - 1) * 2].data then
    min := L.items[2 * n - (i - 1) * 2].data
  else
    min := L.items[i + 1].data;
  
  if min > max then
    max := min;
  end;
  
  writeln(max);
end.
  \end{Verbatim}

  \section*{Вывод}
  В ходе выполнения лабораторной работы удалось освоить и реализовать такую структуру данных как двунаправленный список путем решения предложенной задачи.

\end{document}