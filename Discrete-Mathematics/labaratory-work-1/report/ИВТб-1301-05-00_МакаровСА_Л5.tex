\documentclass[a4paper,14pt]{extarticle}

\usepackage[a4paper,top=20mm,bottom=20mm,left=30mm,right=10mm]{geometry}
\usepackage[T1,T2A]{fontenc}
\usepackage[utf8]{inputenc}
\usepackage[russian]{babel}
\usepackage{indentfirst}
\usepackage{titlesec}
\usepackage{graphicx}
\usepackage{verbatim}
\usepackage{fancyvrb}
\usepackage{nicematrix}

\renewcommand{\baselinestretch}{1.3}
\titleformat{\section}{\normalsize\bfseries}{\thesection}{1em}{}
\titleformat{\subsection}{\normalsize\bfseries}{\thesection}{1em}{}
\setlength{\parindent}{12.5mm}

\begin{document}

	\newpage\thispagestyle{empty}
	\begin{center}
		\MakeUppercase{
			Министерство науки и высшего образования Российской Федерации\\
			Федеральное государственное бюджетное образовательное учреждение высшего образования\\
			<<Вятский Государственный Университет>>\\
		}
		Институт математики и информационных систем\\
		Факультет автоматики и вычислительной техники\\
		Кафедра электронных вычислительных машин
	\end{center}
	\vfill
	
	\begin{center}
		\textbf{Работа с множествами}\\
		Отчёт по лабораторной работе №1\\
		по дисциплине\\
		<<Дискретная математика>>\\
		Вариант 5
	\end{center}
	\vfill
	
	\noindent
	\begin{tabular}{ll}
		Выполнил студент гр. ИВТб-1301-05-00 \hspace{5mm} &
		\rule[-1mm]{25mm}{0.10mm}\,/Макаров С.А./\\
		
		Руководитель преподаватель & \rule[-1mm]{25mm}{0.10mm}\,/Пахарева И.В./\\
	\end{tabular}
	
	\vfill
	\begin{center}
		Киров 2024
	\end{center}
	
	\newpage
	\section*{Цель}
	Цель лабораторной работы: изучение основ теории множеств, базовых операций над ними, разработка приложения на языке Паскаль согласно заданию.
	
	\section*{Задание}
	Требуется реализовать программу для выполнения заданных операций над множествами:
	\begin{enumerate}
		\item Программа должна позволять вводить до десяти с однолитеральными именами (A, B, C и т.д.) за счёт использования с жёстко заданным синтаксисом <имя множества> = <элемент 1>, <элемент 2>, .... Множество может включать в себя до десяти элементов.
		
		\item Программа должна позволять выполнять заданные операции над введёнными множествами за счёт использования строки с жёстко заданным синтаксисом, с выводом результата в качестве работы.
		
		\item В случае некорректного введения строки в пунктах 1 и 2 должно появляться информационное сообщение об ошибке.
	\end{enumerate}
	
	Данные для выполнения задания:\\
	$\cup$ -- объединение множеств\\
	$\cap$ -- пересечение множеств\\
	- -- разность множеств\\
	$\triangle$ -- симметричная разность множеств\\
	N -- множество натуральных чисел\\
	Z -- множество целых чисел\\
	Q -- множество рациональных чисел\\
	I -- множество иррациональных чисел\\
	R -- множество действительных чисел\\
	рус. -- множество букв русского алфавита\\
	лат. -- множество букв латинского алфавита\\
	
	\noindent Таблица 1 -- Вариант задания\\
	\begin{NiceTabularX}{\textwidth}{cXXXXXccc}[hvlines, cell-space-top-limit=1mm]
		вар./множ. & \centering A & \centering B & \centering C & \centering D & \centering E & X & Y & K \\
		
		5 & \centering R & \centering Q & \centering N & \centering лат. & \centering рус. & A$\cap$B$\cap$C & E$\triangle$D & X$\cap$Y
	\end{NiceTabularX}
	
	\section*{Решение}
	
	\section*{Вывод}
	В ходе выполнения лабораторной работы были изучены способы работы с множествами путём выполннения предложенных задач. Также был реализован консольный интерфейс для ввода мощности множеств и ввода самих множеств в соответствии с вариантом задания.

\end{document}