\documentclass[a4paper,14pt]{extarticle}

\usepackage[a4paper,top=20mm,bottom=20mm,left=30mm,right=10mm]{geometry}
\usepackage[T1,T2A]{fontenc}
\usepackage[utf8]{inputenc}
\usepackage[russian]{babel}
\usepackage{indentfirst}
\usepackage{titlesec}
\usepackage{graphicx}
\usepackage{verbatim}
\usepackage{fancyvrb}

\renewcommand{\baselinestretch}{1.3}
\titleformat{\section}{\normalsize\bfseries}{\thesection}{1em}{}
\titleformat{\subsection}{\normalsize\bfseries}{\thesection}{1em}{}
\setlength{\parindent}{12.5mm}

\begin{document}

	\newpage\thispagestyle{empty}
	\begin{center}
		\MakeUppercase{
			Министерство науки и высшего образования Российской Федерации\\
			Федеральное государственное бюджетное образовательное учреждение высшего образования\\
			<<Вятский Государственный Университет>>\\
		}
		Институт математики и информационных систем\\
		Факультет автоматики и вычислительной техники\\
		Кафедра электронных вычислительных машин
	\end{center}
	\vfill
	
	\begin{center}
		Отчет по лабораторной работе №5\\
		по дисциплине\\
		<<Программирование>>\\
	\end{center}
	\vfill
	
	\noindent
	\begin{tabular}{ll}
		Выполнил студент гр. ИВТб-1301-05-00 \hspace{5mm} &
		\rule[-1mm]{25mm}{0.10mm}\,/Макаров С.А./\\
		
		Руководитель зав. кафедры ЭВМ & \rule[-1mm]{25mm}{0.10mm}\,/Долженкова М.Л./\\
	\end{tabular}
	
	\vfill
	\begin{center}
		Киров 2024
	\end{center}
	
	\newpage
	\section*{Цель}
	Цель лабораторной работы: получить базовые сведения о наиболее известных алгоритмах сортировки, изучить принципы работы с текстовыми файлами.
	
	\section*{Задание}
	\begin{enumerate}
		\item Реализовать сортировку данных с помощью алгоритма подсчётом.
		
		\item Реализовать сортировку данных с помощью поразрядного алгоритма.
		
		\item В обоих случаях необходимо предусмотреть возможность изменения компаратора (реализация компаратора в виде передаваемой в подпрограмму функции).
		
		\item Считывание и вывод данных необходимо производить из текстового файла.
		
		\item Для демонстрации работы программных реализаций самостоятельно подготовить варианты входных данных (при этом объем тестовых файлов должен позволять оценить скорость работы программ).
	\end{enumerate}
	
	\pagebreak
	\section*{Решение}
	
	\section*{Вывод}
	В ходе выполнения лабораторной работы были изучены алгоритмы сортировки подсчётом и поразрядной сортировки. Также были изучены принципы работы с текстовыми файлами путём решения предложенных задач.

\end{document}