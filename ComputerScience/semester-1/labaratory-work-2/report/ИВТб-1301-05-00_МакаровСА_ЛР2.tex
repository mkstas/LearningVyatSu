\documentclass[a4paper,14pt]{extarticle}

\usepackage[a4paper,top=20mm,right=20mm,bottom=20mm,left=20mm]{geometry}
\usepackage[T2A]{fontenc}
\usepackage[utf8]{inputenc}
\usepackage[russian]{babel}
\usepackage{indentfirst}
\usepackage{titlesec}

\renewcommand{\baselinestretch}{1.5}
\titleformat{\section}{\normalsize\bfseries}{\thesection}{1em}{}
\titleformat{\subsection}{\normalsize\bfseries}{\thesection}{1em}{}
\setlength{\parindent}{12.5mm}

\begin{document}
	
	\newpage\pagestyle{empty}
	\begin{center}
	\textbf{
		\MakeUppercase{
			Министерство науки и высшего образования Российской Федерации\\
			Федеральное государственное бюджетное образовательное учреждение высшего образования\\
			<<Вятский Государственный Университет>>\\
		}
		Институт математики и информационных систем\\
		Факультет автоматики и вычислительной техники\\
		Кафедра электронных вычислительных машин
	}
	\end{center}
	\vfill
	
	\hfill
	\begin{tabular}{l}
		\footnotesize Дата сдачи на проверку:\\
		\footnotesize <<\rule[-1mm]{5mm}{0.10mm}\/>>\rule[-1mm]{20mm}{0.10mm}\ 2024 г.\\
		\footnotesize Проверено:\\
		\footnotesize <<\rule[-1mm]{5mm}{0.10mm}\/>>\rule[-1mm]{20mm}{0.10mm}\ 2024 г.
	\end{tabular}
	\vfill
	
	\begin{center}
		\textbf{\MakeUppercase{Арифметические операции в системах счисления}}\\
		Отчет по лабораторной работе №2\\
		по дисциплине\\
		<<Информатика>>\\
		Вариант 10
	\end{center}
	\vfill
	
	\noindent
	\begin{tabular}{ll}
		Выполнил студент гр. ИВТб-1301-05-00 &
		\rule[-1mm]{30mm}{0.10mm}\,/Макаров С.А./\\
		& \hspace{8mm}\footnotesize(подпись)\\
		
		Руководитель доцент кафедры ЭВМ & \rule[-1mm]{30mm}{0.10mm}\,/Коржавина А.С./\\
		& \hspace{8mm}\footnotesize(подпись)\\
	\end{tabular}
	
	\noindent
	\begin{tabular}{lp{58mm}r}
		Работа защищена &  & <<\rule[-1mm]{5mm}{0.10mm}\/>>\rule[-1mm]{30mm}{0.10mm}\ 2024 г.
	\end{tabular}
	\vfill
	
	\begin{center}
		Киров 2024
	\end{center}
	
	\newpage
	\section*{Цель}
	Цель лабораторной работы: закрепить на практике знания о выполнении арифметических операций сложения и умножения чисел в позиционных и непозиционных системах счисления.
	
	\section*{Задание}
	\begin{enumerate}
		\item В каждом варианте даны 2 пары чисел (X1 и Y1, X2 и Y2).  Выполнить перевод чисел из десятичной системы счисления в двоичную систему счисления (2СС), выполнить сложение и умножение чисел. Проверить полученные результаты.
		
		\item  В каждом варианте даны 2 пары чисел (X3 и Y3, X4 и Y4). Выполнить перевод чисел из десятичной системы счисления в 16СС.  Выполнить сложение шестнадцатеричных чисел в соответствии с вариантом. Проверить полученный результат.
		
		\item Выполнить перевод в систему остаточных классов в соответствии с вариантом. В каждом варианте даны 2 числа (А и В) и соответствующие им базисы. Выполнить сложение и умножение чисел. Проверить полученный результаты.
		
		\item Выполнить перевод в троичную симметричную систему счисления в соответствии с вариантом. В каждом варианте даны 2 числа. Выполнить сложение чисел. Проверить полученные результаты.
		
		\item Выполнить перевод в двоично-десятичную систему счисления в соответствии с вариантом. В каждом варианте даны 2 пары чисел. Представить первую пару чисел в коде 8421 (код с естественными весами), вторую пару в коде 8421+3. Проверить полученные результаты.
	\end{enumerate}
	
	\section*{Решение}
	Для перевода целой части числа в K-ичную систему счисления необходимо воспользоваться формулой:
	
	\[
		X = K \cdot \Biggl(\sum_{i=0}^{n}b_{i} \cdot K^{i-1}\Biggl)\;+\; b_{0},
	\]
	
	\noindent получая в остатке $b_{0} \in [0, K - 1]$. Далее формула принимает следующий вид:
	
	\[
		X = K \cdot X^{(1)} + b_{0}.
	\]
	
	Для перевода дробной части числа в K-ичную систему счисления необходимо воспользоваться формулой:
	
	\[
		Y \cdot K = K \cdot \Biggl(\sum_{i=-m}^{-1}b_{i} \cdot K^{i}\Biggl) =b_{-1} + \sum_{i=-m}^{-2}b_{i} \cdot K^{i+1},
	\]
	
	находим $b_{-1} = Y \cdot K$ и справедливо $b_{-1} \in [0, K - 1]$. Далее формула принимает следующий вид:
	
	\[
		Y \cdot K = b_{-1} + Y^{1}.
	\]
	
	\subsection*{Задание 1}
	Даны два целых числа в десятичной системе счисления: 121 и 114. Переведем их в двоичную систему счисления:
	
	$121 \equiv 1111001_{2}$
	
	\[
	\begin{array}[c]{ll}
		121 = 60 \cdot 2 + 1, &\Rightarrow b_{0} = 1,\\
		60 = 30 \cdot 2 + 0, &\Rightarrow b_{1} = 0,\\
		30 = 15 \cdot 2 + 0, &\Rightarrow b_{2} = 0,\\
		15 = 7 \cdot 2 + 1, &\Rightarrow b_{3} = 1,\\
		7 = 3 \cdot 2 + 1, &\Rightarrow b_{4} = 1,\\
		3 = 1 \cdot 2 + 1, &\Rightarrow b_{5} = 1,\\
		1 = 0 \cdot 2 + 1, &\Rightarrow b_{6} = 1.
	\end{array}
	\]
	
	$114 \equiv 1110010 $
	
	\[
	\begin{array}[c]{ll}
		114 = 57 \cdot 2 + 0, &\Rightarrow b_{0} = 0,\\
		57 = 28 \cdot 2 + 1, &\Rightarrow b_{1} = 1,\\
		28 = 14 \cdot 2 + 0, &\Rightarrow b_{2} = 0,\\
		14 = 7 \cdot 2 + 1, &\Rightarrow b_{3} = 0,\\
		7 = 3 \cdot 2 + 1,  &\Rightarrow b_{4} = 1,\\
		3 = 1 \cdot 2 + 1,  &\Rightarrow b_{5} = 1,\\
		1 = 0 \cdot 2 + 1,  &\Rightarrow b_{6} = 1.
	\end{array}
	\]
	
	Далее сложим два числа в двоичной системе счисления:
	
	\[
	\begin{tabular}{{c}{c}}
		\texttt{+} &
		\begin{tabular}{c}
			\texttt{~1111001}\\
			\texttt{~1110010}\\
			\hline
		\end{tabular}\\
		& \texttt{11101011}
	\end{tabular}
	\]
	
	Проверим сложение двух двоичных чисел:\\
	$11101011_2=2^{7}+2^{6}+2^{5}+2^{3}+2^{1}+2^{0}=128+64+32+8+2+1=235=121+114$
	
	Умножим два числа в двоичной системе счисления:
	
	\[
	\begin{tabular}{{c}{c}}
		\texttt{$\times$}&
		\begin{tabular}{c}
			\texttt{~~~~~~1111001}\\
			\texttt{~~~~~~1110010}\\
			\hline
		\end{tabular}\\
		& \texttt{~~~~~~0000000}\\
		& \texttt{~~~~~1111001~}\\
		& \texttt{~~~~0000000~~}\\
		& \texttt{~~~0000000~~~}\\
		& \texttt{~~1111001~~~~}\\
		& \texttt{~1111001~~~~~}\\
		& \texttt{1111001~~~~~~}\\
		\hline
		& \texttt{11010111100010}
	\end{tabular}
	\]
	
	Проверим результат умножения:\\
	$11010111100010_{2}=2^{13}+2^{12}+2^{10}+2^{8}+2^{7}+2^{6}+2^{5}+2^{1}=8192+4096++1024+256+128+64+32+2=13794=121 \cdot 114$.
	
	Даны два числа в десятичной системе счисления: 9,8 и 6,8. Переведем их в двоичную систему счисления.
	
	$9,8 \equiv 1001,1100_{2}$
	
	Целая часть:
	
	\[
	\begin{array}[c]{ll}
		9 = 4 \cdot 2 + 1, &\Rightarrow b_{0} = 1,\\
		4 = 2 \cdot 2 + 0, &\Rightarrow b_{1} = 0,\\
		2 = 0 \cdot 2 + 0, &\Rightarrow b_{2} = 0,\\
		1 = 0 \cdot 2 + 1, &\Rightarrow b_{3} = 1.
	\end{array}
	\]
	
	Дробная часть:
	
	\[
	\begin{array}[c]{ll}
		0,8 \cdot 2 = 1 + 0,6 &\Rightarrow b_{-1} = 1,\\
		0,6 \cdot 2 = 1 + 0,2 &\Rightarrow b_{-2} = 1,\\
		0,2 \cdot 2 = 0 + 0,4 &\Rightarrow b_{-3} = 0,\\
		0,4 \cdot 2 = 0 + 0,8 &\Rightarrow b_{-4} = 0.\\
	\end{array}
	\]
	
	$6,8 \equiv 110,1100_{2}$
	
	Целая часть:
	
	\[
	\begin{array}[c]{ll}
		6 = 3 \cdot 2 + 0, &\Rightarrow b_{0} = 0,\\
		3 = 1 \cdot 2 + 1, &\Rightarrow b_{1} = 1,\\
		1 = 0 \cdot 2 + 1, &\Rightarrow b_{2} = 1.
	\end{array}
	\]
	
	Дробная часть:
	
	\[
	\begin{array}[c]{ll}
		0,8 \cdot 2 = 1 + 0,6 &\Rightarrow b_{-1} = 1,\\
		0,6 \cdot 2 = 1 + 0,2 &\Rightarrow b_{-2} = 1,\\
		0,2 \cdot 2 = 0 + 0,4 &\Rightarrow b_{-3} = 0,\\
		0,4 \cdot 2 = 0 + 0,8 &\Rightarrow b_{-4} = 0.\\
	\end{array}
	\]
	
	Далее сложим два числа в двоичной системе счисления:
	
	\[
	\begin{tabular}{{c}{c}}
		\texttt{+} &
		\begin{tabular}{c}
			\texttt{~1001,1100}\\
			\texttt{~~110,1100}\\
			\hline
		\end{tabular}\\
		& \texttt{10000,1000}
	\end{tabular}
	\]
	
	Проверим сложение двух двоичных чисел.\\
	Целая часть: $10000_{2}=2^{4}=16$.\\
	Дробная часть: $1000_{2}=2^{-1}=0,5$.\\
	Полное число: $16+0,5=16,5 \approx 16,6=9,8+6,8$.
	
	Умножим два числа в двоичной системе счисления:
	
	\[
	\begin{tabular}{{c}{c}}
		\texttt{$\times$}&
		\begin{tabular}{c}
			\texttt{~~~~~~~1001,1100}\\
			\texttt{~~~~~~~~110,1100}\\
			\hline
		\end{tabular}\\
		& \texttt{~~~~~~~~00000000}\\
		& \texttt{~~~~~~~00000000~}\\
		& \texttt{~~~~~~10011100~~}\\
		& \texttt{~~~~~10011100~~~}\\
		& \texttt{~~~~00000000~~~~}\\
		& \texttt{~~~10011100~~~~~}\\
		& \texttt{~~10011100~~~~~~}\\
		\hline
		& \texttt{1000001,11010000}
	\end{tabular}
	\]
	
	Проверим умножение:\\
	Целая часть: $1000001_{2}=2^{6}+2^{0}=64+1=65$. \\
	Дробная часть: $11010000_{2}=2^{-1}+2^{-2}+2^{-4}=0,5+0,25+0,0625=0,8125$.
	Полное число: $65+0,8125=65,8125 \approx 66,64=9,8 \cdot 6,8$.
	
	\subsection*{Задание 2}
	
	Даны два числа в десятичной системе счисления: 125 и 101. Переведем их в шестнадцатеричную систему счисления:
	
	$125 \equiv 7D_{16}$
	
	\[
	\begin{array}[c]{ll}
		125 = 7 \cdot 16 + 13, &\Rightarrow b_{0} = D,\\
		7 = 0 \cdot 16 + 7, &\Rightarrow b_{1} = 7,\\
	\end{array}
	\]
	
	\pagebreak
	
	$101 \equiv 65_{16}$
	
	\[
	\begin{array}[c]{ll}
		101 = 6 \cdot 16 + 5, &\Rightarrow b_{0} = 5,\\
		6 = 0 \cdot 16 + 6, &\Rightarrow b_{1} = 6,\\
	\end{array}
	\]
	
	Выполним сложение шестнадцатеричных чисел:
	
	\[
	\begin{tabular}{{c}{c}}
		\texttt{+} &
		\begin{tabular}{c}
			\texttt{7D}\\
			\texttt{65}\\
			\hline
		\end{tabular}\\
		& \texttt{E2}
	\end{tabular}
	\]
	
	Далее проверим полученный результат:\\
	$E2_{16}=E \cdot 16^{1} + 2 \cdot 16^{0}=224+2=226=125+101$.
	
	Даны два числа в десятичной системе счисления: 88 и 82. Переведем их в шестнадцатеричную систему счисления:
	
	$88 \equiv 58_{16}$
	
	\[
	\begin{array}[c]{ll}
		88 = 5 \cdot 16 + 8, &\Rightarrow b_{0} = 8,\\
		5 = 0 \cdot 16 + 5, &\Rightarrow b_{1} = 5,\\
	\end{array}
	\]
	
	$82 \equiv 52_{16}$
	
	\[
	\begin{array}[c]{ll}
		82 = 5 \cdot 16 + 2, &\Rightarrow b_{0} = 2,\\
		5 = 0 \cdot 16 + 6, &\Rightarrow b_{1} = 5.\\
	\end{array}
	\]
	
	Выполним сложение шестнадцатеричных чисел:
	
	\[
	\begin{tabular}{{c}{c}}
		\texttt{+} &
		\begin{tabular}{c}
			\texttt{58}\\
			\texttt{52}\\
			\hline
		\end{tabular}\\
		& \texttt{AA}
	\end{tabular}
	\]
	
	Далее проверим полученный результат:\\
	$AA_{16}=A \cdot 16^{1} + A \cdot 16^{0}=160+10=170=88+82$.
	
	\pagebreak
	
	\subsection*{Задание 3}
	
	Даны числа 86 и 73. Переведем их в систему остаточных классов в соответствии с базисом \{5, 7, 11, 13\}.
	
	$86 \equiv \{1,\;2,\;9,\;8\}$
	
	\[
	\begin{array}[c]{l}
		86\;mod\;5 = 1,\\
		86\;mod\;7 = 2,\\
		86\;mod\;11 = 9,\\
		86\;mod\;13 = 8,\\
	\end{array}
	\]
	
	$73 \equiv \{3,\;3,\;7,\;8\}$
	
	\[
	\begin{array}[c]{l}
		73\;mod\;5 = 3,\\
		73\;mod\;7 = 3,\\
		73\;mod\;11 = 7,\\
		73\;mod\;13 = 8.\\
	\end{array}
	\]
	
	Выполним сложение чисел в системе остаточных классов:\\
	$\{1,\;2,\;9,\;8\}+\{3,\;3,\;7,\;8\}=\{4,\;5,\;5,\;3\}$
	
	\[
	\begin{array}[c]{l}
		(1+3)\;mod\;5 = 4,\\
		(2+3)\;mod\;7 = 5,\\
		(9+7)\;mod\;11 = 5,\\
		(8+8)\;mod\;13 = 3.
	\end{array}
	\]
	
	Проверим полученный результат путем сложения исходных чисел и их переводом в систему остаточных классов:\\
	$86+73=159 \equiv\ \{4,\;5,\;5,\;3\}$
	
	\[
	\begin{array}[c]{l}
		159\;mod\;5 = 4,\\
		159\;mod\;7 = 5,\\
		159\;mod\;11 = 5,\\
		159\;mod\;13 = 3.\\
	\end{array}
	\]
	
	\pagebreak
	
	Далее выполним умножение чисел в системе остаточных классов:\\
	$\{1,\;2,\;9,\;8\} \cdot \{3,\;3,\;7,\;8\}=\{3,\;6,\;8,\;12\}$
	
	\[
	\begin{array}[c]{l}
		(1\cdot3)\;mod\;5 = 3,\\
		(2\cdot3)\;mod\;7 = 6,\\
		(9\cdot7)\;mod\;11 = 8,\\
		(8\cdot8)\;mod\;13 = 12.
	\end{array}
	\]
	
	Проверим полученный результат путем умножения исходных чисел и их переводом в систему остаточных классов:\\
	$86\cdot73=6278 \equiv\ \{3,\;6,\;8,\;12\}$
	
	\[
	\begin{array}[c]{l}
		6278\;mod\;5 = 3,\\
		6278\;mod\;7 = 6,\\
		6278\;mod\;11 = 8,\\
		6278\;mod\;13 = 12.\\
	\end{array}
	\]
	
	\subsection*{Задание 4}
	
	Даны числа 112 и 79. Переведем их в троичную симметричную систему счисления, после чего выполним сложение.
	
	$112 \equiv pp0pp$
	
	\[
	\begin{array}[c]{ll}
		112 = 37 \cdot 3 + 1, &\Rightarrow b_{0} = p,\\
		37 = 12 \cdot 3 + 1, &\Rightarrow b_{1} = p,\\
		12 = 4 \cdot 3 + 0, &\Rightarrow b_{2} = 0,\\
		4 = 1 \cdot 3 + 1, &\Rightarrow b_{3} = p,\\
		1 = 0 \cdot 3 + 1, &\Rightarrow b_{4} = p.
	\end{array}
	\]
	
	\pagebreak
	
	$79 \equiv p00np$
	
	\[
	\begin{array}[c]{ll}
		79 = 26 \cdot 3 + 1, &\Rightarrow b_{0} = p,\\
		26 = 9 \cdot 3 - 1, &\Rightarrow b_{1} = n,\\
		9 = 3 \cdot 3 + 0, &\Rightarrow b_{2} = 0,\\
		3 = 1 \cdot 3 + 0, &\Rightarrow b_{3} = 0,\\
		1 = 0 \cdot 3 + 1, &\Rightarrow b_{4} = p.
	\end{array}
	\]
	
	Далее выполним сложение чисел:
	
	\[
	\begin{tabular}{{c}{c}}
		\texttt{+} &
		\begin{tabular}{c}
			\texttt{~pp0pp}\\
			\texttt{~p00np}\\
			\hline
		\end{tabular}\\
		& \texttt{pnp0pn}
	\end{tabular}
	\]
	
	Проверим результат сложения чисел:\\
	$pnp0pn=3^5-3^4+3^3+3^1-3^0=243-81+27+3-1=191=112+79$.
	
	\subsection*{Задание 5}
	Даны два числа в десятичной системе счисления: 125 и 103. Переведем в соответствии с их двоичным представлением в коде <<8421>>. При необходимости проведем коррекцию результата.
	
	$125 \equiv 0001\;0010\;0101$,\\
	\indent $103 \equiv 0001\;0000\;0011$.
	
	Выполним сложение чисел:
	
	\[
	\begin{tabular}{{c}{c}}
		\texttt{+} &
		\begin{tabular}{c}
			\texttt{0001~0010~0101}\\
			\texttt{0001~0000~0011}\\
			\hline
		\end{tabular}\\
		& \texttt{0010~0010~1000}
	\end{tabular}
	\]
	
	Проверим полученный результат: $0010\;0010\;1000=228=125+103$.
	
	Даны два числа: 117 и 82. Переведем в соответствии с их двоичным представлением в коде <<8421+3>>. При необходимости проведем коррекцию результата.
	
	$117 \equiv 0100\;0100\;1010$,\\
	\indent $82 \equiv 1011\;0101$.
	
	Выполним сложение чисел:
	
	\[
	\begin{tabular}{{c}{c}}
		\texttt{+} &
		\begin{tabular}{c}
			\texttt{0100~0100~1010}\\
			\texttt{0011~1011~0101}\\
			\hline
		\end{tabular}\\
		\texttt{+} &
		\begin{tabular}{c}
			\texttt{0111~1111~1111}\\
			\texttt{1101~1101~1101}\\
			\hline
		\end{tabular}\\
		& \texttt{0100~1100~1100}
	\end{tabular}
	\]
	
	Проверим полученный результат: $0100\;1100\;1100=199=117+82$.
	
	\section*{Вывод}
	В ходе выполнения лабораторной работы освоены сложение и умножение чисел в двоичной и шестнадцатеричной системе счисления. Также было выполнен перевод чисел в систему остаточных классов, их сложение и умножение. Помимо этого был реализован перевод чисел в троичною симметричную систему счисления и их сложение. Кроме того был изучено и проработано сложение чисел в кодах <<8421>> и <<8421+3>>.
	
\end{document}
