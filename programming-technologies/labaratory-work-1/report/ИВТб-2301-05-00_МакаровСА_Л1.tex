\documentclass[a4paper,14pt]{extarticle}

\usepackage[a4paper,top=20mm,bottom=20mm,left=30mm,right=10mm]{geometry}
\usepackage[T1,T2A]{fontenc}
\usepackage[utf8]{inputenc}
\usepackage[russian]{babel}
\usepackage{indentfirst}
\usepackage{titlesec}
\usepackage{graphicx}
\usepackage{verbatim}
\usepackage{fancyvrb}

\renewcommand{\baselinestretch}{1.3}
\titleformat{\section}{\normalsize\bfseries}{\thesection}{1em}{}
\titleformat{\subsection}{\normalsize\bfseries}{\thesection}{1em}{}
\setlength{\parindent}{12.5mm}

\begin{document}

  \newpage\thispagestyle{empty}
  \begin{center}
    \MakeUppercase{
      Министерство науки и высшего образования Российской Федерации\\
      Федеральное государственное бюджетное образовательное учреждение высшего образования\\
      <<Вятский Государственный Университет>>\\
    }
    Институт математики и информационных систем\\
    Факультет автоматики и вычислительной техники\\
    Кафедра электронных вычислительных машин
  \end{center}
  \vfill

  \begin{center}
    Отчет по лабораторной работе №1\\
    по дисциплине\\
    <<Технологии программирования>>\\
  \end{center}
  \vfill

  \noindent
  \begin{tabular}{ll}
    Выполнил студент гр. ИВТб-2301-05-00 \hspace{5mm} &
    \rule[-1mm]{25mm}{0.10mm}\,/Макаров С.А./\\
    
    Проверил преподаватель & \rule[-1mm]{25mm}{0.10mm}\,/Пащенко Д.Э./\\
  \end{tabular}

  \vfill
  \begin{center}
    Киров 2025
  \end{center}

  \newpage
  \section*{Цель}
  Цель: Приобрести практические навыки создания и публикации веб-сайта с использованием конструктора Tilda.

  \section*{Задание}
  \begin{itemize}
    \item[--] Изучить теоретический материал
    \item[--] Согласовать тему создаваемого сайта с преподавателем и внести её в таблицу
    \item[--] Разработать сайт по приложенным методическим материалам
    \item[--] Опубликовать сайт и приложить его в отчёт
    \item[--] Сделать отчёт в latex, загрузить его
  \end{itemize}

  \section*{Решение}
  Тема созданного сайта - Лэндинг умного дома. Согласно заданию на сайте были созданы 3 веб-страниц, которые содержали в себе: главную страницу, страница устройств, страница скачивания мобильного приложения.

  \pagebreak
  \begin{figure}[h]
    \centering
    \includegraphics[width=0.76\linewidth]{img/1.png}
  \end{figure}
  \begin{center}
    Рисунок 1 – Главная страница
  \end{center}

  \pagebreak
  \begin{figure}[h]
    \centering
    \includegraphics[width=0.76\linewidth]{img/2.png}
  \end{figure}
  \begin{center}
    Рисунок 2 – Страница умных устройств
  \end{center}

  \begin{figure}[h]
    \centering
    \includegraphics[width=0.76\linewidth]{img/3.png}
  \end{figure}
  \begin{center}
    Рисунок 3 – Страница скачивания мобильного приложения
  \end{center}

  \pagebreak
  \section*{Вывод}
  В ходе выполнения лабораторной работы разработан веб-сайт в конструкторе Tilda согласно выбранной и согласованной предметной области.

\end{document}