\documentclass[a4paper,14pt]{extarticle}

\usepackage[a4paper,top=20mm,bottom=20mm,left=30mm,right=10mm]{geometry}
\usepackage[T1,T2A]{fontenc}
\usepackage[utf8]{inputenc}
\usepackage[russian]{babel}
\usepackage{indentfirst}
\usepackage{titlesec}
\usepackage{graphicx}
\usepackage{verbatim}
\usepackage{fancyvrb}

\renewcommand{\baselinestretch}{1.3}
\titleformat{\section}{\normalsize\bfseries}{\thesection}{1em}{}
\titleformat{\subsection}{\normalsize\bfseries}{\thesection}{1em}{}
\setlength{\parindent}{12.5mm}

\begin{document}

  \newpage\thispagestyle{empty}
  \begin{center}
    \MakeUppercase{
      Министерство науки и высшего образования Российской Федерации\\
      Федеральное государственное бюджетное образовательное учреждение высшего образования\\
      <<Вятский Государственный Университет>>\\
    }
    Институт математики и информационных систем\\
    Факультет автоматики и вычислительной техники\\
    Кафедра электронных вычислительных машин
  \end{center}
  \vfill

  \begin{center}
    Отчет по лабораторной работе №5\\
    по дисциплине\\
    <<Технологии программирования>>\\
  \end{center}
  \vfill

  \noindent
  \begin{tabular}{ll}
    Выполнил студент гр. ИВТб-2301-05-00 \hspace{5mm} &
    \rule[-1mm]{25mm}{0.10mm}\,/Макаров С.А./\\
    
    Проверил преподаватель & \rule[-1mm]{25mm}{0.10mm}\,/Пащенко Д.Э./\\
  \end{tabular}

  \vfill
  \begin{center}
    Киров 2025
  \end{center}

  \newpage
  \section*{Цель}
  Цель: Ознакомиться с принципами компоновки и научиться компоновать стандартные виджеты фреймворка Flutter.

  \section*{Задание}
  Измените цвет AppBar используя класс Colors. В тело экрана добавьте виджет текста содержащий название дисциплины и примените к нему произвольные стили TextStyle. Добавить в тело экрана виджет Column. В виджет Column добавить 5–6 контейнеров задать в которых содержаться картинки из ассетов. Каждому контейнеру задать отступ.

  \pagebreak
  \section*{Решение}
  
  В ходе выполнения заданий разработано мобильное приложение, экранная форма которого представлена на рисунке 1.

  \begin{figure}[h]
    \centering
    \includegraphics[width=0.5\linewidth]{img/image.png}
  \end{figure}
  \begin{center}
    Рисунок 3 – Экранная форма мобильного приложения
  \end{center}

  \pagebreak
  Исходный код корневого компонента представлен ниже:

  \noindent
  \begin{Verbatim}[tabsize=4,fontsize=\small]
import 'package:app/pages/home_page.dart';
import 'package:flutter/material.dart';

void main() {
  runApp(const App());
}

class App extends StatelessWidget {
  const App({super.key});

  @override
  Widget build(BuildContext context) => const MaterialApp(
    home: HomePage(),
  );
}
  \end{Verbatim}

  Исходный код главной страницы представлен ниже:

  \noindent
  \begin{Verbatim}[tabsize=4,fontsize=\small]
import 'package:flutter/material.dart';

class HomePage extends StatelessWidget {
  const HomePage({super.key});

  @override
  Widget build(BuildContext context) => Scaffold(
    backgroundColor: Colors.black45,
    appBar: AppBar(
      title: const Text(
        'Макаров Станислав Алексеевич',
        style: TextStyle(
          color: Colors.white,
        ),
      ),
      backgroundColor: Colors.blueAccent,
    ),
    body: Padding(
      padding: const EdgeInsets.all(12.0),
      child: ListView(
        children: [
          const SizedBox(height: 12),
          const Text(
            'Технологии программирования',
            style: TextStyle(
              fontSize: 24,
              fontWeight: FontWeight.w500,
              color: Colors.white,
            ),
          ),
          const SizedBox(height: 24),
          Container(
            margin: const EdgeInsets.only(bottom: 16),
            decoration: const BoxDecoration(
              borderRadius: BorderRadius.all(Radius.circular(12)),
              image: DecorationImage(
                image: AssetImage('assets/image1.jpg'),
                fit: BoxFit.cover,
              ),
            ),
            height: 200,
          ),
          Container(
            margin: const EdgeInsets.only(bottom: 16),
            decoration: const BoxDecoration(
              borderRadius: BorderRadius.all(Radius.circular(12)),
              image: DecorationImage(
                image: AssetImage('assets/image2.jpg'),
                fit: BoxFit.cover,
              ),
            ),
            height: 200,
          ),
          Container(
            margin: const EdgeInsets.only(bottom: 16),
            decoration: const BoxDecoration(
              borderRadius: BorderRadius.all(Radius.circular(12)),
              image: DecorationImage(
                image: AssetImage('assets/image3.jpg'),
                fit: BoxFit.cover,
              ),
            ),
            height: 200,
          ),
          Container(
            margin: const EdgeInsets.only(bottom: 16),
            decoration: const BoxDecoration(
              borderRadius: BorderRadius.all(Radius.circular(12)),
              image: DecorationImage(
                image: AssetImage('assets/image4.jpg'),
                fit: BoxFit.cover,
              ),
            ),
            height: 200,
          ),
          Container(
            margin: const EdgeInsets.only(bottom: 16),
            decoration: const BoxDecoration(
              borderRadius: BorderRadius.all(Radius.circular(12)),
              image: DecorationImage(
                image: AssetImage('assets/image5.jpg'),
                fit: BoxFit.cover,
              ),
            ),
            height: 200,
          ),
        ],
      ),
    ),
  );
}
  \end{Verbatim}

  \section*{Вывод}
  В ходе выполнения лабораторной работы освоены принципы компановки стандартных виджетов фреймворка Flutter. Установлена среда разработки Android Studio, а также Flutter SDK. Путем выполения заданий разработано мобильное приложение в соответствии с вариантом, запущенное с помощью эмулятора.

\end{document}