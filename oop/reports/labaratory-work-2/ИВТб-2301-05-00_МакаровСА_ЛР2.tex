\documentclass[a4paper,14pt]{extarticle}

\usepackage[a4paper,top=20mm,bottom=20mm,left=30mm,right=10mm]{geometry}
\usepackage[T1,T2A]{fontenc}
\usepackage[utf8]{inputenc}
\usepackage[russian]{babel}
\usepackage{indentfirst}
\usepackage{titlesec}
\usepackage{graphicx}
\usepackage{verbatim}
\usepackage{fancyvrb}

\renewcommand{\baselinestretch}{1.3}

\titleformat{\section}{\normalsize\bfseries}{\thesection}{1em}{}
\titleformat{\subsection}{\normalsize\bfseries}{\thesubsection}{1em}{}

\setlength{\parindent}{12.5mm}

\begin{document}

\newpage\thispagestyle{empty}
\begin{center}
  \MakeUppercase{
    Министерство науки и высшего образования Российской Федерации\\
    Федеральное государственное бюджетное образовательное учреждение высшего образования\\
    <<Вятский Государственный Университет>>\\
  }
  Институт математики и информационных систем\\
  Факультет автоматики и вычислительной техники\\
  Кафедра электронных вычислительных машин
\end{center}
\vfill

\begin{center}
  Отчет по лабораторной работе №2\\
  по дисциплине\\
  <<Объектно-ориентированное программирование>>\\
\end{center}
\vfill

\noindent
\begin{tabular}{ll}
  Выполнил студент гр. ИВТб-2301-05-00 \hspace{5mm} & \rule[-1mm]{25mm}{0.10mm}\,/Макаров С.А./ \\
  Руководитель преподаватель                        & \rule[-1mm]{25mm}{0.10mm}\,/Шмакова Н.А./ \\
\end{tabular}

\vfill
\begin{center}
  Киров 2026
\end{center}

\newpage
\section*{\hspace{12.5mm}Цель работы}

\section*{\hspace{12.5mm}Задание}

\section*{\hspace{12.5mm}Решение}

\section*{\hspace{12.5mm}Вывод}

\end{document}