\documentclass[a4paper,14pt]{extarticle}

\usepackage[a4paper,top=20mm,bottom=20mm,left=30mm,right=10mm]{geometry}
\usepackage[T1,T2A]{fontenc}
\usepackage[utf8]{inputenc}
\usepackage[russian]{babel}
\usepackage{indentfirst}
\usepackage{titlesec}
\usepackage{graphicx}
\usepackage{verbatim}
\usepackage{fancyvrb}
\usepackage{hyperref}

\renewcommand{\baselinestretch}{1.3}

\titleformat{\section}{\normalsize\bfseries}{\thesection}{1em}{}
\titleformat{\subsection}{\normalsize\bfseries}{\thesubsection}{1em}{}

\setlength{\parindent}{12.5mm}

\begin{document}

\newpage\thispagestyle{empty}
\begin{center}
  \MakeUppercase{
    Министерство науки и высшего образования Российской Федерации\\
    Федеральное государственное бюджетное образовательное учреждение высшего образования\\
    <<Вятский Государственный Университет>>\\
  }
  Институт математики и информационных систем\\
  Факультет автоматики и вычислительной техники\\
  Кафедра электронных вычислительных машин
\end{center}
\vfill

\begin{center}
  Отчет по лабораторной работе №1\\
  по дисциплине\\
  <<Объектно-ориентированное программирование>>\\
\end{center}
\vfill

\noindent
\begin{tabular}{ll}
  Выполнил студент гр. ИВТб-2301-05-00 \hspace{5mm} & \rule[-1mm]{25mm}{0.10mm}\,/Макаров С.А./ \\
  Руководитель преподаватель                        & \rule[-1mm]{25mm}{0.10mm}\,/Шмакова Н.А./ \\
\end{tabular}

\vfill
\begin{center}
  Киров 2026
\end{center}

\newpage
\section*{\hspace{12.5mm}Цель работы}
Цель работы: изучить и освоить принципы объектно-ориентированного программирования, включая инкапсуляцию, наследование и полиморфизм, путем создания иерархии классов в выбранной предметной области.

\section*{\hspace{12.5mm}Задание}
\begin{enumerate}
  \item Создать иерархию классов состоящую не менее чем из одного родительского  и двух дочерних классов.
  \item В каждом классе определить не менее  двух член данных. не менее двух собственных, а для дочерних не менее двух унаследованных и двух перекрытых член функций.
  \item Разработать приложение демонстрирующее принципы инкапсуляции, наследования и полиморфизма.
\end{enumerate}

\section*{\hspace{12.5mm}Решение}
Предметная область описывает управление ассортиментом блюд в заведении общественного питания. Система позволяет хранить информацию о различных блюдах, учитывать их специфические характеристики, рассчитывать итоговую цену для клиента в зависимости от этих характеристик и выводить информацию о блюде в удобном виде.

Схема структуры иерархии классов представлена в виде диаграммы классов, представленая на рисунке 1.

\pagebreak
\begin{figure}[ht]
  \centering
  \includegraphics[width=1\linewidth]{img/uml.png}
\end{figure}
\begin{center}
  Рисунок 1 – Диаграмма классов
\end{center}

Описание меотодов классов:
\begin{itemize}
  \item[--] Класс Dish:
        \begin{itemize}
          \item[--] Dish() -- конструктор, инициализирует название блюда, вес в граммах и базовую цену;
          \item[--] Dish(string name) -- конструктор, инициализирует название блюда, вес в граммах и базовую цену;
          \item[--] Dish(string name, int weight, double price = 15.0) -- конструктор, инициализирует название блюда, вес в граммах и базовую цену;
          \item[--] string GetName() const -- возвращает название блюда;
          \item[--] int GetWeight() const -- возвращает вес блюда в граммах;
          \item[--] double GetPrice() const -- возвращает базовую цену блюда;
          \item[--] virtual void DisplayInfo() = 0 -- виртуальный метод отображения информации;
          \item[] virtual double GetFullPrice() = 0 -- виртуальный метод, возвращающий полную цену;
        \end{itemize}
  \item[--] Класс Pizza наследуется от Dish:
        \begin{itemize}
          \item[--] Pizza() -- конструктор, передаёт параметры в базовый класс Dish, тип теста по умолчанию — Thick;
          \item[--] Pizza(string name) -- конструктор, передаёт параметры в базовый класс Dish, тип теста по умолчанию — Thick;
          \item[--] Pizza(string name, int weight, double price) -- конструктор, передаёт параметры в базовый класс Dish, тип теста по умолчанию — Thick;
          \item[--] string GetDough() const -- возвращает текстовое описание текущего типа теста;
          \item[--] void ChangeDough() -- переключает тип теста между Thin и Thick;
          \item[--] void CutInSlices() -- имитирует действие «нарезать пиццу на куски», выводит соответствующее сообщение;
          \item[--] void DisplayInfo() override -- переопределённый метод, выводит информацию о пицце: название, тип теста, вес, итоговая цена;
          \item[--] double GetFullPrice() override -- возвращает итоговую цену с учётом типа теста;
        \end{itemize}
  \item[--] Класс Salad наследуется от Dish:
        \begin{itemize}
          \item[--] Salad() -- конструктор, передаёт параметры в базовый класс Dish, заправка по умолчанию — OliveOil;
          \item[--] Salad(string name) -- конструктор, передаёт параметры в базовый класс Dish, заправка по умолчанию — OliveOil;
          \item[--] Salad(string name, int weight, double price) -- конструктор, передаёт параметры в базовый класс Dish, заправка по умолчанию — OliveOil;
          \item[--] string GetDressing() const -- возвращает текстовое описание текущей заправки;
          \item[--] void ChangeDressing() -- переключает вид заправки между OliveOil и Mayonnaise;
          \item[--] void TossWithDressing() const -- bмитирует действие «перемешать салат с заправкой», выводит соответствующее сообщение;
          \item[--] void DisplayInfo() override -- gереопределённый метод, выводит информацию о салате: название, вид заправки, вес, цена;
          \item[--] double GetFullPrice() override -- возвращает итоговую цену с учётом заправки.
        \end{itemize}
\end{itemize}

Исходный код программы находится в репозитории GitHub:

\url{https://github.com/mkstas/VyatSu/tree/master/oop/labaratories/labaratory-work-1}

\section*{\hspace{12.5mm}Вывод}
В ходе выполнения лабораторной работы была разработана и реализована иерархия классов, моделирующая управление ассортиментом блюд в заведении общественного питания. Изучены основные принципы объектно-ориентированного программированя: инкапсуляция, наследование, полиморфизм. Разработано приложение, демонстрирующее применение данных принципов.

\end{document}