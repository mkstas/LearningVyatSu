\documentclass[a4paper,14pt]{extarticle}

\usepackage[a4paper,top=10mm,right=10mm,bottom=10mm,left=10mm]{geometry}
\usepackage[T2A]{fontenc}
\usepackage[utf8]{inputenc}
\usepackage[russian]{babel}

\renewcommand{\baselinestretch}{1.3}

\begin{document}
	\pagestyle{empty}
	\noindent Макаров С.А. ИВТб-1301-05-00 \\
	Ценности Российского мировоззрения \\
	
	Семья является одной из важнейших ценностей в жизни человека. Семья представляет собой группу людей, связанные по родству, общими интересами, традициями. Семья служит для формирования у человека нравственных ценностей о передачи их от поколения к поколению.
	
	Данная ценность значима поскольку в семье воспитание и развитие человека как личности. Внутри семьи, в большинстве случаях, человек учиться уважать и заботиться о других людях. В семье учишься общаться и договариваться с людьми. Кроме того семья является источником финансовой поддержки, оказывает помощь в различных трудных ситуациях. Таким образом для общества ценна тем, что позволяет сохранять культурные традиции, передавая следующим поколениям, укрепляет социальные связи в обществе и поднимает его культурность и доброжелательные отношения, что в свою ведет к снижению конфликтов между людьми, преступности бедности и социальной изоляции людей.
	
	В российской действительности данная ценность проявляется в виде поддержки семьи через законодательство и социальные программы. Существует множество программ на поддержку семей, такие как материнский капитал, льготы для поддержки многодетных семей, поддержка малоимущих, ежемесячное выплаты для детей до 3 лет и т.д. Помимо этого развиваются и организуются различные мероприятия, праздников на укрепление семейных ценностей и отношений между родственниками. Строятся детские площадки, центры для семейного отдыха.
	
	Подводя итоги можно утверждать, что семья является одной из важнейших ценностей, которая играет огромную роль в жизни каждого человека и для общества в целом. На уровне государства происходит поддержка семьи путем различных социальных льгот, создание комфортных условий для воспитания детей.
\end{document}