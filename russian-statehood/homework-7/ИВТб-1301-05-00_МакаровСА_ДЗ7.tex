\documentclass[a4paper,14pt]{extarticle}

\usepackage[a4paper,top=10mm,right=10mm,bottom=10mm,left=10mm]{geometry}
\usepackage[T2A]{fontenc}
\usepackage[utf8]{inputenc}
\usepackage[russian]{babel}
\usepackage{nicematrix}

\renewcommand{\baselinestretch}{1.1}

\begin{document}
	\pagestyle{empty}
	\noindent Макаров С.А. ИВТб-1301-05-00\\
	Поправки в Конституции 2020 г.\\
	
	Поправки в Конституцию вступили в силу 4 июля 2020 года. Список изменений состоит из 206 поправок. Были закреплены меры социальной поддержки, верховенство российского права, новые требования к президенту, неприкосновенность для президента, прекратившего исполнение своих полномочий, закрепление роли Госсовета, новые требования к чиновникам, депутатам и судьям, усиление роли Госдумы, Совета Федерации, изменение численности судей Конституционного Суда с 19 до 11, защита исторической правды и будущего нашей страны.\\
	
	Поправки в главе 4. \\
	
	\noindent
	\begin{NiceTabular}{p{92mm}p{91mm}}[hvlines, cell-space-limits=3mm]
		\Block{1-1}{Было} & \Block{1-1}{Стало} \\
		
		Статья 80: \newline
		2. Президент Российской Федерации является гарантом Конституции Российской Федерации, прав и свобод человека и гражданина. В установленном Конституцией Российской Федерации порядке он принимает меры по охране суверенитета Российской Федерации, ее независимости и государственной целостности, обеспечивает согласованное функционирование и взаимодействие органов государственной власти. &
		Статья 80:\newline
		2. Президент Российской Федерации является гарантом Конституции Российской Федерации, прав и свобод человека и гражданина. В установленном Конституцией Российской Федерации порядке он принимает меры по охране суверенитета Российской Федерации, ее независимости и государственной целостности, \textcolor{red}{поддерживает
		гражданский мир и согласие в стране,} обеспечивает согласованное функционирование и взаимодействие \textcolor{red}{органов, входящих в единую систему публичной власти государственной власти.} \\
		
		Статья 81: \newline
		2. Президентом Российской Федерации может быть избран гражданин Российской Федерации не моложе 35 лет, постоянно проживающий в Российской Федерации не менее 10 лет. \newline 3. Одно и то же лицо не может занимать должность Президента Российской Федерации более двух сроков \textcolor{red}{подряд.} &
		2. Президентом Российской Федерации может быть избран гражданин Российской Федерации не моложе 35 лет, постоянно проживающий в Российской Федерации не менее \textcolor{red}{25 лет, не имеющий и не имевший ранее гражданства иностранного государства либо вида на жительство или иного документа, подтверждающего право на постоянное проживание гражданина Российской Федерации на территории иностранного государства. Требование к кандидату на должность}
	\end{NiceTabular}
	
	\pagebreak
	\noindent
	\begin{NiceTabular}{p{92mm}p{91mm}}[hvlines, cell-space-limits=3mm]
		& \textcolor{red}{Президента Российской Федерации об отсутствии у него гражданства иностранного государства не распространяется на граждан Российской Федерации, ранее имевших гражданство государства, которое было принято или часть которого была принята в Российскую Федерацию в соответствии с федеральным конституционным законом, и постоянно проживавших на территории принятого в Российскую Федерацию государства или территории принятой в Российскую Федерацию части государства. Президенту Российской Федерации в порядке, установленном федеральным законом, запрещается открывать и иметь счета (вклады), хранить наличные денежные средства и ценности в иностранных банках, расположенных за пределами территории} \newline
		3. Одно и то же лицо не может занимать
		должность Президента Российской Федерации более
		двух сроков. \newline
		\textcolor{black}{
			$3^1$. Положение части 3 статьи 81 Конституции Российской Федерации, ограничивающее число сроков, в течение которых одно и то же лицо может занимать должность Президента Российской Федерации, применяется к лицу, занимавшему и (или) занимающему должность Президента Российской Федерации, без учета числа сроков, в течение которых оно занимало и (или) занимает эту должность на момент вступления в силу поправки к Конституции Российской Федерации, вносящей соответствующее ограничение, и не исключает для него возможность занимать должность Президента Российской Федерации в течение сроков, допустимых указанным положением.
		} \\
	\end{NiceTabular}
	
	\pagebreak
	\noindent
	\begin{NiceTabular}{p{92mm}p{91mm}}[hvlines, cell-space-limits=3mm]
		Статья 82: \newline
		2. Присяга приносится в торжественной обстановке в присутствии членов Совета Федерации, депутатов Государственной Думы и судей Конституционного Суда Российской Федерации. &
		Статья 82: \newline
		2. Присяга приносится в торжественной обстановке в присутствии \textcolor{red}{сенаторов Российской Федерации }, депутатов Государственной Думы и судей Конституционного Суда Российской Федерации. \\
		
		Статья 83: \newline
		Президент Российской Федерации: \newline
		a) назначает \textcolor{red}{с согласия Государственной Думы} Председателя Правительства Российской Федерации; \newline
		б) \textcolor{red}{имеет право} председательствовать на заседаниях Правительства Российской Федерации; \newline
		д) \textcolor{red}{по предложению Председателя Правительства Российской Федерации} назначает на должность \textcolor{red}{и освобождает от должности} заместителей Председателя Правительства Российской Федерации, федеральных министров; \newline
		е) представляет Совету Федерации кандидатуры для назначения на должность судей Конституционного Суда Российской Федерации, Верховного Суда Российской Федерации; назначает судей других федеральных судов; \newline
		е$^1$) представляет Совету Федерации кандидатуры для назначения на должность Генерального прокурора Российской Федерации и заместителей Генерального прокурора Российской Федерации; вносит в Совет Федерации предложения об освобождении от должности Генерального прокурора Российской Федерации и заместителей Генерального прокурора Российской Федерации; назначает на должность и освобождает от должности прокуроров субъектов Российской Федерации, а также иных прокуроров, кроме прокуроров городов, районов и приравненных к ним прокуроров;
		&
		Статья 83: \newline
		a) назначает Председателя Правительства Российской Федерации,
		\textcolor{red}{кандидатура которого утверждена Государственной Думой по представлению Президента Российской Федерации, и освобождает Председателя Правительства Российской Федерации от должности;} \newline
		б) \textcolor{red}{осуществляет общее руководство Правительством Российской Федерации; вправе }председательствовать на заседаниях Правительства Российской Федерации; \newline
		\textcolor{black}{
			б$^1$) утверждает по предложению Председателя Правительства Российской Федерации структуру федеральных органов исполнительной власти, вносит в нее изменения; в структуре федеральных органов исполнительной власти определяет органы, руководство деятельностью которых осуществляет Президент Российской Федерации, и органы, руководство деятельностью которых осуществляет Правительство Российской Федерации. В случае если Председатель Правительства Российской Федерации освобожден Президентом Российской Федерации от должности, вновь назначенный Председатель Правительства Российской Федерации не представляет Президенту Российской Федерации предложения о структуре федеральных органов исполнительной власти;
		}
	\end{NiceTabular}
	
	\pagebreak
	\noindent
	\begin{NiceTabular}{p{92mm}p{91mm}}[hvlines, cell-space-limits=3mm]
	ж) формирует и возглавляет Совет Безопасности Российской Федерации, статус которого определяется федеральным законом;
	и) формирует Администрацию Президента Российской Федерации;
	& \textcolor{black}{
		в$^1$) принимает отставку Председателя Правительства Российской Федерации, заместителей Председателя Правительства Российской Федерации, федеральных министров, а также руководителей федеральных органов исполнительной власти, руководство деятельностью которых осуществляет Президент Российской Федерации;
	} \newline
	д) назначает на должность заместителей Председателя Правительства Российской Федерации и федеральных министров, \textcolor{red}{кандидатуры которых утверждены Государственной Думой, (за исключением федеральных министров, указанных в пункте «д1» настоящей статьи), и освобождает их от должности;} \newline
	\textcolor{black}{
		д$^1$) назначает на должность после консультаций с Советом Федерации и освобождает от должности руководителей федеральных органов исполнительной власти (включая федеральных министров), ведающих вопросами обороны, безопасности государства, внутренних дел, юстиции, иностранных дел, предотвращения чрезвычайных ситуаций и ликвидации последствий стихийных бедствий, общественной безопасности;
	} \newline
	е) представляет Совету Федерации кандидатуры для назначения на должность \textcolor{red}{Председателя Конституционного Суда Российской Федерации, заместителя	Председателя Конституционного Суда Российской Федерации и судей Конституционного Суда Российской Федерации, Председателя Верховного Суда Российской Федерации, заместителей Председателя Верховного Суда Российской Федерации и судей Верховного Суда Российской Федерации; назначает председателей, заместителей председателей и судей других федеральных судов;}
	\end{NiceTabular}
	
	\pagebreak
	\noindent
	\begin{NiceTabular}{p{92mm}p{91mm}}[hvlines, cell-space-limits=3mm]
		& е$^1$) \textcolor{black}{назначает на должность после консультаций с Советом Федерации и освобождает от должности Генерального прокурора Российской Федерации, заместителей Генерального прокурора Российской Федерации, прокуроров субъектов Российской Федерации, прокуроров военных и других специализированных прокуратур, приравненных к прокурорам субъектов Российской Федерации; назначает на должность и освобождает от должности иных прокуроров, для которых такой порядок назначения и освобождения от должности установлен федеральным законом;} \newline
		\textcolor{black}{е$^3$) вносит в Совет Федерации представление о прекращении в соответствии с федеральным конституционным законом полномочий Председателя Конституционного Суда Российской Федерации, заместителя Председателя Конституционного Суда Российской Федерации и судей Конституционного Суда Российской Федерации, Председателя Верховного Суда Российской Федерации, заместителей Председателя Верховного Суда Российской Федерации и судей Верховного Суда Российской Федерации, председателей, заместителей председателей и судей кассационных и апелляционных судов в случае совершения ими поступка, порочащего честь и достоинство судьи, а также в иных предусмотренных федеральным конституционным законом случаях, свидетельствующих о невозможности осуществления судьей своих полномочий;} \newline
		\textcolor{black}{е$^4$) представляет Совету Федерации кандидатуры для назначения на должность Председателя Счетной палаты и половины от общего числа аудиторов Счетной}
	\end{NiceTabular}
	
	\pagebreak
	\noindent
	\begin{NiceTabular}{p{92mm}p{91mm}}[hvlines, cell-space-limits=3mm]
		& \textcolor{black}{палаты; представляет Государственной Думе кандидатуры для назначения на должность заместителя Председателя Счетной палаты и половины от общего числа аудиторов Счетной палаты;} \newline
		\textcolor{black}{е$^5$) формирует Государственный Совет Российской Федерации в целях обеспечения согласованного функционирования и взаимодействия органов публичной власти, определения основных направлений внутренней и внешней политики Российской Федерации и приоритетных направлений социальноэкономического развития государства; статус Государственного Совета Российской Федерации определяется федеральным законом;} \newline
		ж) \textcolor{black}{формирует Совет Безопасности Российской Федерации в целях содействия главе государства в реализации его полномочий по вопросам обеспечения национальных интересов и безопасности личности, общества и государства, а также поддержания гражданского мира и согласия в стране, охраны суверенитета Российской Федерации, ее независимости и государственной целостности, предотвращения внутренних и внешних угроз; возглавляет Совет Безопасности Российской Федерации. Статус Совета Безопасности Российской Федерации определяется федеральным законом;} \newline
		и) формирует Администрацию Президента Российской Федерации \textcolor{red}{в целях обеспечения реализации своих полномочий;} \\
		Статья $92^1$: \newline
		отсутствует &
		Статья $92^1$: \newline
		1. Президент Российской Федерации, прекративший исполнение полномочий в связи с истечением срока его пребывания в должности либо досрочно в случае его отставки или стойкой неспособности по состоянию здоровья осуществлять принадлежащие ему полномочия,
	\end{NiceTabular}
	
	\pagebreak
	\noindent
	\begin{NiceTabular}{p{92mm}p{91mm}}[hvlines, cell-space-limits=3mm]
		& обладает неприкосновенностью. \newline
		2. Иные гарантии Президенту Российской Федерации, прекратившему исполнение полномочий в связи с истечением срока его пребывания в должности либо досрочно в случае его отставки или стойкой неспособности по состоянию здоровья осуществлять принадлежащие ему полномочия, устанавливаются федеральным законом. \newline
		3. Президент Российской Федерации, прекративший исполнение своих полномочий, может быть лишен неприкосновенности в порядке, предусмотренном статьей 93 Конституции Российской Федерации. \\
		Статья 93: \newline
		1. Президент Российской Федерации может быть отрешен от должности Советом Федерации только на основании выдвинутого Государственной Думой обвинения в государственной измене или совершении иного тяжкого преступления, подтвержденного заключением Верховного Суда Российской Федерации о наличии в действиях Президента Российской Федерации признаков преступления и заключением Конституционного Суда Российской Федерации о соблюдении установленного порядка выдвижения обвинения. \newline
		2. Решение Государственной Думы о выдвижении обвинения и решение Совета Федерации об отрешении Президента от должности должны быть приняты двумя третями голосов от общего числа в каждой из палат по инициативе не менее одной трети депутатов Государственной Думы и при наличии заключения специальной комиссии, образованной Государственной Думой.
		&
		Статья 93: \newline
		1. Президент Российской Федерации может быть отрешен от должности \textcolor{red}{а Президент Российской Федерации, прекративший исполнение своих полномочий, лишен неприкосновенности}, Советом Федерации только на основании выдвинутого Государственной Думой обвинения в государственной измене или совершении иного тяжкого преступления, подтвержденного заключением Верховного Суда Российской Федерации о наличии в действиях Президента Российской Федерации \textcolor{red}{и, как действующего, так и прекратившего исполнение своих полномочий,} признаков преступления и заключением Конституционного Суда Российской Федерации о соблюдении установленного порядка выдвижения обвинения. \newline
		2. Решение Государственной Думы о выдвижении обвинения и решение Совета Федерации об отрешении Президента \textcolor{red}{Российской Федерации} от должности\textcolor{red}{, о}
	\end{NiceTabular}
	
	\pagebreak
	\noindent
	\begin{NiceTabular}{p{92mm}p{91mm}}[hvlines, cell-space-limits=3mm]
		3. Решение Совета Федерации об отрешении Президента Российской Федерации от должности должно быть принято не позднее чем в трехмесячный срок после выдвижения Государственной Думой обвинения против Президента. Если в этот срок решение Совета Федерации не будет принято, обвинение против Президента считается отклоненным.
		&
		\textcolor{red}{лишении неприкосновенности Президента Российской Федерации, прекратившего исполнение своих полномочий,} должны быть приняты двумя третями голосов от общего числа \textcolor{red}{соответственно сенаторов Российской Федерации и депутатов Государственной Думы} по инициативе не менее одной трети депутатов Государственной Думы и при наличии заключения специальной комиссии, образованной Государственной Думой. \newline
		3. Решение Совета Федерации об отрешении Президента Российской Федерации от должности\textcolor{red}{, о лишении неприкосновенности Президента Российской Федерации, прекратившего исполнение своих полномочий,} должно быть принято не позднее чем в трехмесячный срок после выдвижения Государственной Думой обвинения против Президента \textcolor{red}{Российской Федерации} Если в этот срок решение Совета Федерации не будет принято, обвинение против Президента \textcolor{red}{ Российской Федерации, Президента Российской Федерации, прекратившего исполнение своих полномочий,} считается отклоненным.
	\end{NiceTabular}
	
\end{document}