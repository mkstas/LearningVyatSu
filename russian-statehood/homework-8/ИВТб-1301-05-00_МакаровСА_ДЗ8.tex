\documentclass[a4paper,14pt]{extarticle}

\usepackage[a4paper,top=10mm,right=10mm,bottom=10mm,left=10mm]{geometry}
\usepackage[T2A]{fontenc}
\usepackage[utf8]{inputenc}
\usepackage[russian]{babel}

\renewcommand{\baselinestretch}{1.3}

\begin{document}
	\pagestyle{empty}
	\noindent Макаров С.А. ИВТб-1301-05-00\\
	Традиционные ценности\\
	
	\begin{enumerate}
		\item Традиционные ценности -- это нравственные ориентиры, формирующие мировоззрение граждан России, передаваемые от поколения к поколению, лежащие в основе общероссийской гражданской идентичности и единого культурного пространства страны, укрепляющие гражданское единство, нашедшие свое уникальное, самобытное проявление в духовном, историческом и культурном развитии многонационального народа России.\\
		
		К традиционным ценностям относятся жизнь, достоинство, права и свободы человека, патриотизм, гражданственность, служение Отечеству и ответственность за его судьбу, высокие нравственные идеалы, крепкая семья, созидательный труд, приоритет духовного над материальным, гуманизм, милосердие, справедливость, коллективизм, взаимопомощь и взаимоуважение, историческая память и преемственность поколений, единство народов России.
		
		\item Угрозу традиционным ценностям представляют деятельность экстремистских и террористических организаций, отдельных средств массовой информации и массовых коммуникаций, действия недружественных иностранных государств, ряда транснациональных корпораций и иностранных некоммерческих организаций, деятельность некоторых организаций и лиц на территории России.\\

		Распространение деструктивной идеологии влечет за собой следующие риски:
		\begin{itemize}
			\item[--] создание условий для саморазрушения общества, ослабление семейных, дружеских и иных социальных связей;
			
			\item[--] усиление социокультурного расслоения общества, снижение роли социального партнерства, обесценивание идей созидательного труда и взаимопомощи;
			
			\item[--] причинение вреда нравственному здоровью людей, навязывание представлений, предполагающих отрицание человеческого достоинства и ценности человеческой жизни;
			
			\item[--] внедрение антиобщественных стереотипов поведения, распространение аморального образа жизни, вседозволенности и насилия, рост употребления алкоголя и наркотиков;
			
			\item[--] формирование общества, пренебрегающего духовно-нравственными ценностями;
			
			\item[--] искажение исторической правды, разрушение исторической памяти;
			
			\item[--] отрицание российской самобытности, ослабление общероссийской гражданской идентичности и единства многонационального народа России, создание условий для межнациональных и межрелигиозных конфликтов;
			
			\item[--] подрыв доверия к институтам государства, дискредитация идеи служения Отечеству, формирование негативного отношения к воинской службе и государственной службе в целом.
		\end{itemize}
		
		Также к угрозам можно отнести навязывание в социальных сетях неважности образования для достижения достойного уровня жизни, что влечет понижение количества квалифицированных специалистов во всех сферах общественной жизни, отсутствие ответственных кадров на руководящих должностях, что ведет к неразумному использованию государственного бюджету, неправильному определению сфер, приоритетных для развития в регионах страны. Помимо перечисленного отсутствие должного образования ведет к отсутствию критического мышления, что ведет к повышенной подверженности деструктивной идеологии.
		
		\item Для решения искажения исторической правды и переписывания истории необходимо развивать исторические музеи, места, привлекать к посещению их молодежь путем предоставления льготных билетов и организации бесплатных мероприятий. Помимо этого необходимо повышать квалификацию преподавателей истории, использование достоверных учебников в образовательных учреждений, рассказывать об исторических искажениях в СМИ, сети интернет, социальных сетях.
	\end{enumerate}
	
\end{document}