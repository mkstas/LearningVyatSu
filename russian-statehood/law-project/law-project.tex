\documentclass[a4paper,14pt]{extarticle}

\usepackage[a4paper,top=10mm,right=20mm,bottom=10mm,left=20mm]{geometry}
\usepackage[T2A]{fontenc}
\usepackage[utf8]{inputenc}
\usepackage[russian]{babel}
\usepackage{indentfirst}
\usepackage{titlesec}

\renewcommand{\baselinestretch}{1.3}
\titleformat{\section}{\normalsize\bfseries\centering}{\thesection}{1em}{}
\titleformat{\subsection}{\normalsize\bfseries}{\thesection}{1em}{}
\setlength{\parindent}{12.5mm}

\begin{document}
	\pagestyle{empty}
	\section*{Законопроект о возвращении\\ Кировского времени в Кировской области}
	
	\subsection*{Цель законопроекта}
	Обеспечение социальной и экономической гармонизации времени в Кировской области путем возвращения к местному <<Кировскому времени>> (UTC+4), более соответствующему природным биоритмам населения и улучшению условий для деловой и социальной активности.
	
	\subsection*{Основные положения}
	Установить, что Кировская область переходит на время, опережающее московское на 1 час (UTC+4).
	Предлагается вступление закона в силу с 1 января 2026 года.
	
	\subsection*{Обоснование}
	Исследования показали, что смещение времени ближе к астрономическому увеличивает продуктивность населения, снижает уровень стресса и улучшает качество сна.
	Экономическая активность будет более синхронизирована с ритмами большинства населения.
	
	\subsection*{Ожидаемые эффекты}
	\begin{itemize}
		\item[--] Сокращение количества заболеваний, связанных с нарушением сна (до 15\% в год).
		
		\item[--] Увеличение производительности труда на 3-5\%.
		
		\item[--] Повышение туристической привлекательности региона благодаря увеличению светлого времени в вечерние часы.
	\end{itemize}
	
	\subsection*{Техническая реализация}
		\begin{itemize}
		\item[--] Внесение изменений в законодательство РФ о времени (подача предложения в Государственную Думу).
		
		\item[--] Перевод административных, транспортных и цифровых систем на новое время.
	\end{itemize}
	
	\subsection*{Категории населения}
	Рассчитан на все категории населения, проживающие в Кировской области
	
	\subsection*{Бюджет реализации}
	\begin{itemize}
		\item[--] Общий объем затрат: 50 миллионов рублей.
		
		\item[--] Информационная кампания: освещение изменений в СМИ, создание видеороликов, постеров и социальных сетей — 20 млн рублей.
		
		\item[--] Корректировка IT-систем: адаптация программного обеспечения в государственных учреждениях, банках и других организациях — 30 млн рублей.
	\end{itemize}
	
	\noindent Законопроект разработали:\\
	\noindent
	\begin{tabular}{l}
		\rule[-1mm]{30mm}{0.10mm}\, / Пестова А.А.\\
		\rule[-1mm]{30mm}{0.10mm}\, / Елпашев С.А.\\
		\rule[-1mm]{30mm}{0.10mm}\, / Макаров С.А.\\
		\rule[-1mm]{30mm}{0.10mm}\, / Зинин В.А.
	\end{tabular}
\end{document}