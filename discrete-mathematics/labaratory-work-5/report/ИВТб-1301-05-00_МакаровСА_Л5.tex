\documentclass[a4paper,14pt]{extarticle}

\usepackage[a4paper,top=20mm,bottom=20mm,left=30mm,right=10mm]{geometry}
\usepackage[T1,T2A]{fontenc}
\usepackage[utf8]{inputenc}
\usepackage[russian]{babel}
\usepackage{indentfirst}
\usepackage{titlesec}
\usepackage{graphicx}
\usepackage{verbatim}
\usepackage{fancyvrb}

\renewcommand{\baselinestretch}{1.3}
\setlength{\parindent}{12.5mm}
\titleformat{\section}{\normalsize\bfseries}{\indent\thesection}{1em}{}
\titleformat{\subsection}{\normalsize\bfseries}{\indent\thesubsection}{1em}{}

\begin{document}

  \newpage\thispagestyle{empty}
  \begin{center}
    \MakeUppercase{
      Министерство науки и высшего образования Российской Федерации\\
      Федеральное государственное бюджетное образовательное учреждение высшего образования\\
      <<Вятский Государственный Университет>>\\
    }
    Институт математики и информационных систем\\
    Факультет автоматики и вычислительной техники\\
    Кафедра электронных вычислительных машин
  \end{center}
  \vfill

  \begin{center}
    \textbf{Связность графов}\\
    Отчёт по лабораторной работе №5\\
    по дисциплине\\
    <<Дискретная математика>>\\
    Вариант 8
  \end{center}
  \vfill

  \noindent
  \begin{tabular}{ll}
    Выполнил студент гр. ИВТб-1301-05-00 \hspace{5mm} &
    \rule[-1mm]{25mm}{0.10mm}\,/Макаров С.А./\\
    
    Руководитель преподаватель & \rule[-1mm]{25mm}{0.10mm}\,/Пахарева И.В./\\
  \end{tabular}

  \vfill
  \begin{center}
    Киров 2025
  \end{center}

  \newpage
  \section*{\hspace{12.5mm}Цель}
  Цель лабораторной работы: изучение основ теории графов, базовых операций над ними, разработка приложения на языке Паскаль или СИ согласно заданию.

  \section*{\hspace{12.5mm}Задание}
  Граф задан матрицей инцидентности в файле (вершин >= 4, дуг >= 4). Сформировать матрицу связности. Определить является ли граф несвязным.

  \section*{\hspace{12.5mm}Решение}

  \pagebreak
  При разработке реализована программа, исходный код которой представлен ниже.

  \begingroup
    \linespread{1}

    \begin{Verbatim}[tabsize=2]

    \end{Verbatim}
  \endgroup

  \section*{\hspace{12.5mm}Вывод}
  В процессе выполнения лабораторной работы, при решении предложенных задач, реализована программа на языке Паскаль, которая формирует матрицу связности, определяет является ли граф несвязным.

\end{document}