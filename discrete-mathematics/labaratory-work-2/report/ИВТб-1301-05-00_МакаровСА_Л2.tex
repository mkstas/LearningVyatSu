\documentclass[a4paper,14pt]{extarticle}

\usepackage[a4paper,top=20mm,bottom=20mm,left=30mm,right=10mm]{geometry}
\usepackage[T1,T2A]{fontenc}
\usepackage[utf8]{inputenc}
\usepackage[russian]{babel}
\usepackage{indentfirst}
\usepackage{titlesec}
\usepackage{graphicx}
\usepackage{verbatim}
\usepackage{fancyvrb}
\usepackage{nicematrix}

\renewcommand{\baselinestretch}{1.3}
\setlength{\parindent}{12.5mm}
\titleformat{\section}{\normalsize\bfseries}{\indent\thesection}{1em}{}
\titleformat{\subsection}{\normalsize\bfseries}{\indent\thesubsection}{1em}{}

\begin{document}

  \newpage\thispagestyle{empty}
  \begin{center}
    \MakeUppercase{
      Министерство науки и высшего образования Российской Федерации\\
      Федеральное государственное бюджетное образовательное учреждение высшего образования\\
      <<Вятский Государственный Университет>>\\
    }
    Институт математики и информационных систем\\
    Факультет автоматики и вычислительной техники\\
    Кафедра электронных вычислительных машин
  \end{center}
  \vfill
  
  \begin{center}
    \textbf{Работа с графами}\\
    Отчёт по лабораторной работе №2\\
    по дисциплине\\
    <<Дискретная математика>>\\
    Вариант 8
  \end{center}
  \vfill
  
  \noindent
  \begin{tabular}{ll}
    Выполнил студент гр. ИВТб-1301-05-00 \hspace{5mm} &
    \rule[-1mm]{25mm}{0.10mm}\,/Макаров С.А./\\
    
    Руководитель преподаватель & \rule[-1mm]{25mm}{0.10mm}\,/Пахарева И.В./\\
  \end{tabular}
  
  \vfill
  \begin{center}
    Киров 2025
  \end{center}
  
  \newpage
  \section*{Цель}
  Цель лабораторной работы: изучение основ теории графов, базовых операций над ними, разработка приложения на языке Паскаль или СИ согласно заданию.
  
  \section*{Задание}
  Орграф задается матрицей смежности, которая формируется случайным образом. Размерность 4 <= n <= 10 вводится с клавиатуры. Найти номер вершины, имеющей максимальную полустепень исхода, вывести множество соответсвующих дуг найденной вершины.
  
  \section*{Решение}
  
  \section*{Вывод}
  В процессе выполнения лабораторной работы, при решении предложенных задач, изучены операции над множествами и реализованы такие операции как пересечение, симметричная разность, мощность множества. Для решения реализована программа на языке Паскаль, представляющая из себя консольный интерфейс, которая выводит результат решения задач.

\end{document}