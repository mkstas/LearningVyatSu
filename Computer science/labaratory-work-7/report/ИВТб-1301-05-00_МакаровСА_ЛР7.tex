\documentclass[a4paper,14pt]{extarticle}

\usepackage[a4paper,top=20mm,bottom=20mm,left=30mm,right=10mm]{geometry}
\usepackage[T1,T2A]{fontenc}
\usepackage[utf8]{inputenc}
\usepackage[russian]{babel}
\usepackage{indentfirst}
\usepackage{titlesec}
\usepackage{graphicx}

\renewcommand{\baselinestretch}{1.3}
\titleformat{\section}{\normalsize\bfseries}{\thesection}{1em}{}
\titleformat{\subsection}{\normalsize\bfseries}{\thesection}{1em}{}
\setlength{\parindent}{12.5mm}

\begin{document}
	
	\newpage\thispagestyle{empty}
	\begin{center}
		\MakeUppercase{
			Министерство науки и высшего образования Российской Федерации\\
			Федеральное государственное бюджетное образовательное учреждение высшего образования\\
			<<Вятский Государственный Университет>>\\
		}
		Институт математики и информационных систем\\
		Факультет автоматики и вычислительной техники\\
		Кафедра электронных вычислительных машин
	\end{center}
	\vfill
	
	\begin{center}
		Отчет по лабораторной работе №7\\
		по дисциплине\\
		<<Информатика>>\\
		<<Построение комбинационных схем>>\\
		Вариант 10
	\end{center}
	\vfill
	
	\noindent
	\begin{tabular}{ll}
		Выполнил студент гр. ИВТб-1301-05-00 \hspace{5mm} &
		\rule[-1mm]{25mm}{0.10mm}\,/Макаров С.А./\\
		
		Руководитель доцент кафедры ЭВМ & \rule[-1mm]{25mm}{0.10mm}\,/Коржавина А.С./\\
	\end{tabular}
	
	\vfill
	\begin{center}
		Киров 2024
	\end{center}
	
	\newpage
	\section*{Цель}
	Цель лабораторной работы: закрепить на практике знания о минимизации системы булевых функций и получить навыки реализации простейших арифметических устройств.
	
	\section*{Задание}
	\begin{enumerate}
		\item Выполнить минимизацию булевых функций, представить функции различных базисах – основном логическом базисе (И, ИЛИ, НЕ) или в базисе Шеффера (И-НЕ) в соответствии с вариантом, после чего построить схему в системе Logisim и выполнить проверку.
		
		\item Построить четырехразрядный полный сумматор, складывающий 2 двоичных четырехразрядных числа и учитывающий единицу переноса. Построить схему сумматора в Logisim, проверить его работоспособность.
		
		\item Построить четырехразрядный умножитель, перемножающий 2 двоичных четырехразрядных числа. Построить схему умножителя в Logisim, проверить его работоспособность.  Допускается использование следующих логических элементов: И, ИЛИ, НЕ, И-НЕ, ИЛИ-НЕ,  сложение по модулю 2, эквивалентность.
		
		\item Построить 16-разрядный сумматор со схемами ускоренного переноса.  Построить схему сумматора в Logisim, проверить его работоспособность.  Допускается использование следующих логических элементов: И, ИЛИ, НЕ, И-НЕ, ИЛИ-НЕ,  сложение по модулю 2, эквивалентность.
	\end{enumerate}
	
	\newpage
	\section*{Решение}
	\subsection*{Задание 1}
	
	\subsection*{Задание 2}
	
	\subsection*{Задание 3}
	
	\subsection*{Задание 4}
	
	\section*{Вывод}
	
\end{document}