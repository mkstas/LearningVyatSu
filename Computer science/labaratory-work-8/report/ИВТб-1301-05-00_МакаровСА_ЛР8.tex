\documentclass[a4paper,14pt]{extarticle}

\usepackage[a4paper,top=20mm,bottom=20mm,left=30mm,right=10mm]{geometry}
\usepackage[T1,T2A]{fontenc}
\usepackage[utf8]{inputenc}
\usepackage[russian]{babel}
\usepackage{indentfirst}
\usepackage{titlesec}
\usepackage{graphicx}

\renewcommand{\baselinestretch}{1.3}
\titleformat{\section}{\normalsize\bfseries}{\thesection}{1em}{}
\titleformat{\subsection}{\normalsize\bfseries}{\thesection}{1em}{}
\setlength{\parindent}{12.5mm}

\begin{document}
	
	\newpage\thispagestyle{empty}
	\begin{center}
		\MakeUppercase{
			Министерство науки и высшего образования Российской Федерации\\
			Федеральное государственное бюджетное образовательное учреждение высшего образования\\
			<<Вятский Государственный Университет>>\\
		}
		Институт математики и информационных систем\\
		Факультет автоматики и вычислительной техники\\
		Кафедра электронных вычислительных машин
	\end{center}
	\vfill
	
	\begin{center}
		Отчет по лабораторной работе №7\\
		по дисциплине\\
		<<Информатика>>\\
		<<Разработка последовательных схем (счетчиков)>>
	\end{center}
	\vfill
	
	\noindent
	\begin{tabular}{ll}
		Выполнил студент гр. ИВТб-1301-05-00 \hspace{5mm} &
		\rule[-1mm]{25mm}{0.10mm}\,/Макаров С.А./\\
		
		Руководитель доцент кафедры ЭВМ & \rule[-1mm]{25mm}{0.10mm}\,/Коржавина А.С./\\
	\end{tabular}
	
	\vfill
	\begin{center}
		Киров 2024
	\end{center}
	
	\newpage
	\section*{Цель}
	Цель лабораторной работы: закрепить на практике знания о минимизации системы булевых функций и получить навыки реализации простейших арифметических устройств.
	
	\section*{Задание}
	\begin{enumerate}
		\item Построить схемы прямого (на +1) и обратного (на -1) 4-разрядных двоичных счетчиков на счетных (T) триггерах. Построить схемы счетчиков в Logisim, проверить их работоспособность.
		
		\item Построить схему прямого счетчика по модулю 10, то есть считающего в прямом направлении от 0 до 9 на счетных (T) триггерах. Построить схему счетчика в Logisim, проверить его работоспособность.
		
		\item Построить схему прямого счетчика по произвольному модулю N, то есть считающего в прямом направлении от 0 до N-1 на счетных (T) триггерах. Построить схему счетчика в Logisim, проверить его работоспособность.
		
		\item Построить схему прямого счетчика по произвольному модулю N, то есть считающего в прямом направлении от 0 до N-1 на D триггерах. Построить схему счетчика в Logisim, проверить его работоспособность.
		
		\item Построить схему прямого счетчика на +3. Счетчик увеличивает значение на +3, то есть счет идет 0 3 6 9 12 15 0 и т.д. на T триггерах. Построить схему счетчика в Logisim, проверить его работоспособность.
	\end{enumerate}
	
	\newpage
	\section*{Решение}
	\subsection*{Задание 1}

	\subsection*{Задание 2}
	
	\subsection*{Задание 3}
	
	\subsection*{Задание 4}
	
	\subsection*{Задание 5}
	
	\section*{Вывод}
	
\end{document}