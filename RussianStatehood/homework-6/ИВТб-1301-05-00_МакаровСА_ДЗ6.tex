\documentclass[a4paper,14pt]{extarticle}

\usepackage[a4paper,top=10mm,right=10mm,bottom=10mm,left=10mm]{geometry}
\usepackage[T2A]{fontenc}
\usepackage[utf8]{inputenc}
\usepackage[russian]{babel}

\renewcommand{\baselinestretch}{1.3}

\begin{document}
	\pagestyle{empty}
	\noindent Макаров С.А. ИВТб-1301-05-00\\
	Энциклопедическая справка <<Православие, самодержавие, народность>> \\
	
	<<Православие, самодержавие, народность>> -- государственная идеология Российской империи в период царствования Николая I. Автором является министр народного просвещения Сергей Уваров. Данная концепция была разработана в 1833 году.
	
	В основе концепции лежали крайне консервативные, реакционные взгляды на просвещение, науку, литературу. Основная идея в том, что самодержавие является единственно верной формой управления, которая олицетворяла сильную  власть, православие обозначалось, как духовное основание государства, так как его исповедовало большинство населения, а народность подразумевалась как союз между народом и императорской властью, а также закрытость от мыслей и веяний из Европы. Возникла данная концепция из-за опасности по отношению к государственной власти от революционных настроений. В первую очередь из-за восстания декабристом, на идеи которых повлиял Заграничный поход, вследствие знакомства с общественной жизнью Франции, в том числе и Европы.
	
	В дальнейшем данная концепция повлияла на общественный строй России. Россия шла своим уникальным путем развития, защищенным от революционных настроений с Запада. В образовательных учреждениях начали продвигать вышеописанные идеи и подавлять революционные идеи. Помимо этого из-за этой теории отвергались попытки смягчения и отмены крепостного права, начатые Александром I из-за чего оно просуществовало еще до 1861 года. Также сохранялась самодержавная власть, что повлияло на революционные настроения к 1917 году, так как она уже не отвечала современности и уже не удавалось оградить Россию от западных идей о свободах. Повлияло и на дальнейшие периода государства, например, у власти все также оставался ограниченный круг лиц.
\end{document}