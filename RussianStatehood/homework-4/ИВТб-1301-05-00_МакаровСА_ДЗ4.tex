\documentclass[a4paper,14pt]{extarticle}

\usepackage[a4paper,top=10mm,right=10mm,bottom=10mm,left=10mm]{geometry}
\usepackage[T2A]{fontenc}
\usepackage[utf8]{inputenc}
\usepackage[russian]{babel}

\renewcommand{\baselinestretch}{1.3}

\begin{document}
	\pagestyle{empty}
	\noindent Макаров С.А. ИВТб-1301-05-00 \\
	Герой земляк (Оплеснин Н. В.) \\
	
	Во время Великой Отечественной войны было немало солдат из Республики Коми, некоторые из них стали героями Советского Союза. Одним из них был Николай Васильевич Оплеснин.
	
	Николай Васильевич Оплеснин родился 12 декабря 1914 года в селе Выльгорт ныне Сыктывдинского района Республики Коми в семье крестьянина. Жил в Сыктывкаре.
	
	В 1936 году вступил в Красную Армию, где окончил ускоренные курса командиров. В 1939-1940 годах был командиром пулеметного взвода 68-го запасного стрелкового полка и 468-го стрелкового полка 111-й стрелковой дивизии.
	
	На фронтах Великой Отечественной войны с июня 1941 года. Участвовал в боях на Северо-Западном, Ленинградском и Волховском фронтах, являясь офицером разведки штаба стрелкового полка, затем помощником начальника оперативного отделения штаба 111-й дивизии
	
	В сентябре 1941 года дивизия, в которой он служил, попала в окружение. Как командиру разведроты Оплеснину было поручено связаться с частями Красной Армии. С героизмом, выполняя приказ, он с риском для жизни просочился сквозь боевые порядки врага, трижды переплывал холодную реку Волхов. В дальнейшем, при его активном участии, дивизии удалось вырваться из окружения, пересекла реку Волхов и соединилась с войсками Красной армии. Затем, 27 декабря 1941 года был удостоен званием героя Советского Союза, а также были вручены орден Ленина и медаль <<Золотая Звезда>>.
	
	К сожалению Николай Васильевич Оплеснин погиб при подготовке операции в районе города Чудово Новгородской области. Похоронен в городе Чудово в городском парке имени 1 мая. 
	
	В память о Герое названы улицы в городах и районах Республики Коми, а также улица в городе Чудово близ которого погиб Оплеснин.
	
	Героем можно назвать храброго, бескорыстного человека, который готов рискнуть жизнью, свободой или чем-либо еще ценным для помощи или спасения кого-либо. Также героем можно назвать того, кто готов полностью себя отдать для блага общества.
\end{document}