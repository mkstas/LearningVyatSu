\documentclass[a4paper,14pt]{extarticle}

\usepackage[a4paper,top=10mm,right=10mm,bottom=10mm,left=10mm]{geometry}
\usepackage[T2A]{fontenc}
\usepackage[utf8]{inputenc}
\usepackage[russian]{babel}
\usepackage{nicematrix}

\renewcommand{\baselinestretch}{1.3}

\begin{document}
	\pagestyle{empty}
	\noindent Макаров С.А. ИВТб-1301-05-00 \\
	Исторический кризис \\
	
	\noindent
	\begin{NiceTabular}{p{50mm}p{133mm}}[hvlines, cell-space-limits=1mm]
		Краткое описание кризисной ситуации &
		Великая Отечественная война 1941-1945 гг. 22 июня Германия вторглась в Совесткий Союз. Первые годы войны СССР по сути, вел противостояние с Германией и ее союзниками в одиночку. \\
		
		Угрозы кризиса, негативные последствия &
		Угроза уничтожения СССР как государства и страны, уничтожение и эксплуатация славянского населения и меньшинств, проживающих в стране. Последствиями стали значительное сокращение населения, уничтожение предприятий и городов в Европейской части страны. \\
		
		Достижения, связанные с преодолением кризиса &
		Сплочение народа в результате борьбы с Германией, расширение влияния СССР в Европе, а также появление новых коммунистических стран. Государство стало постоянным членом Совета Безопасности ООН и стало обладать правом вето. Также были созданы современные виды вооружения и получен колоссальный опыт ведения боевых действий.
	\end{NiceTabular}
\end{document}