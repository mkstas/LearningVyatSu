\documentclass[a4paper,14pt]{extarticle}
%\documentclass[a4paper,12pt,twocolumn]{extarticle}
%\documentclass[12pt]{article}

\usepackage[a4paper,text={180mm,260mm},left=15mm,top=15mm]{geometry}
\usepackage[utf8]{inputenc}
\usepackage{cmap}
\usepackage[russian]{babel}
\usepackage{textcomp}
\usepackage{indentfirst}
\usepackage{amssymb}
\usepackage{amsmath}
\usepackage{graphicx}

%\begin{document}
%	Текст документа
%\end{document}

\begin{document}
	\title{Заголовок}
	\author{Автор, или перечень авторов}
	\date{Дата создания документа}
	\date\today
	\begin{abstract}
		Аннотация
	\end{abstract}
	\maketitle
	
	$$\delta_{ij}=\begin{cases}1,&i=j\\0,&i\ne j\end{cases}$$
	
	$$\delta_{ij} =
		\begin{cases}
			1, & i=j,\\
			0, & i\ne j
		\end{cases}$$
	
	Текст % А
	будет не % комментарий
	прерывен! % пропадет!
	
	\AA
	\aa
	\textbf\AA
	
	\rule[5pt]{25pt}{1pt}
	\rule[-5pt]{1pt}{10pt}
	\rule{6pt}{6pt}
	
	$x^2_ij$
	$x^2_{ij}$
	
	{\itshape курсив}
	
	\begin{itshape}
		Для верстки большого объема текста лучше пользоваться окружениями
	\end{itshape}
	
	\textbf{при}мер
	
	{Для начала создадим группу, вложенную в обычный текст. \it\large Здесь изменим шрифт на курсив большого размера. {\sf\small Создадим вложенную группу и установим рубленный шрифт малого размера.} Выйдем из вложенной группы. Восстановился большой курсив.} А теперь, выйдя из первой группы, возвратимся к исходному шрифту.
	
	\section[О друзьях]{Слоны мои друзья!}\label{s:слоны}
	
	\appendix
	\section{Название}\label{}
		Текст приложения\dots
	\subsection{Заголовок}\label{}
		Продолжение\dots

	\tableofcontents
	%\thepage
	
	Текст\footnotemark[1]
	\footnotetext[1]{Текст}
	
\end{document}