\documentclass[a4paper, 14pt]{extarticle}

\usepackage[a4paper,top=20mm,bottom=20mm,left=30mm,right=10mm]{geometry}
\usepackage[T1,T2A]{fontenc}
\usepackage[utf8]{inputenc}
\usepackage[russian]{babel}
\usepackage{indentfirst}
\usepackage{titlesec}
\usepackage{verbatim}

\renewcommand{\baselinestretch}{1.3}

\titleformat{\section}{\normalsize\bfseries}{\thesection}{1em}{}
\titleformat{\subsection}{\normalsize\bfseries}{\thesubsection}{1em}{}

\setlength{\parindent}{12.5mm}

\begin{document}

  \newpage\thispagestyle{empty}
  \begin{center}
    \MakeUppercase{
      Министерство науки и высшего образования Российской Федерации\\
      Федеральное государственное бюджетное образовательное учреждение высшего образования\\
      <<Вятский Государственный Университет>>\\
    }
    Институт математики и информационных систем\\
    Факультет автоматики и вычислительной техники\\
    Кафедра электронных вычислительных машин
  \end{center}
  \vfill
  
  \begin{center}
    Отчет по лабораторной работе №1\\
    по дисциплине\\
    <<Информатика>>\\
    <<Светодидные индикаторы>>\\
    Вариант 15
  \end{center}
  \vfill
  
  \noindent
  \begin{tabular}{ll}
    Выполнил студент гр. ИВТб-1301-05-00 \hspace{5mm} &
    \rule[-1mm]{25mm}{0.10mm}\,/Макаров С.А./\\
    
    Руководитель преподаватель & \rule[-1mm]{25mm}{0.10mm}\,/Шмакова Н.А./\\
  \end{tabular}
  
  \vfill
  \begin{center}
    Киров 2025
  \end{center}

  \newpage
  \section*{\hspace{12.5mm}Цель работы}
  Цель работы: ознакомление с элементами Arduino IDE, изучение основ
  работы со средой для программирования, а также сборка схем со
  светодиодными индикаторами.

  \section*{\hspace{12.5mm}Задание}

  \section*{\hspace{12.5mm}Решение}

  \section*{\hspace{12.5mm}Вывод}

\end{document}