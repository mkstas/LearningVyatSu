\documentclass[a4paper,14pt]{extarticle}

\usepackage[a4paper,top=20mm,bottom=20mm,left=30mm,right=10mm]{geometry}
\usepackage[T1,T2A]{fontenc}
\usepackage[utf8]{inputenc}
\usepackage[russian]{babel}
\usepackage{indentfirst}
\usepackage{titlesec}
\usepackage{graphicx}
\usepackage{verbatim}
\usepackage{fancyvrb}

\renewcommand{\baselinestretch}{1.3}
\titleformat{\section}{\normalsize\bfseries}{\thesection}{1em}{}
\titleformat{\subsection}{\normalsize\bfseries}{\thesection}{1em}{}
\setlength{\parindent}{12.5mm}

\begin{document}
	
	\newpage\thispagestyle{empty}
	\begin{center}
		\MakeUppercase{
			Министерство науки и высшего образования Российской Федерации\\
			Федеральное государственное бюджетное образовательное учреждение высшего образования\\
			<<Вятский Государственный Университет>>\\
		}
		Институт математики и информационных систем\\
		Факультет автоматики и вычислительной техники\\
		Кафедра электронных вычислительных машин
	\end{center}
	\vfill
	
	\begin{center}
		Отчет по лабораторной работе №3\\
		по дисциплине\\
		<<Программирование>>\\
	\end{center}
	\vfill
	
	\noindent
	\begin{tabular}{ll}
		Выполнил студент гр. ИВТб-1301-05-00 \hspace{5mm} &
		\rule[-1mm]{25mm}{0.10mm}\,/Макаров С.А./\\
		
		Руководитель зав. кафедры ЭВМ & \rule[-1mm]{25mm}{0.10mm}\,/Долженкова М.Л./\\
	\end{tabular}
	
	\vfill
	\begin{center}
		Киров 2024
	\end{center}
	
	\newpage
	\section*{Цель}
	Цель лабораторной работы: освоить синтаксис построения процедур и функций, изучить способы передачи данных в подпрограммы, получить навыки организации минимального пользовательского интерфейса.
	
	\section*{Задание}
	Реализовать программу вычисления площади фигуры, ограниченной кривой
	$2 * x^3 -2 * x^2 + 0 * x + 16$ и осью OX (в положительной части по оси OY). Вычисление определенного интеграла должно выполняться численно, с применением метода левых прямоугольников. Пределы интегрирования вводятся пользователем. Взаимодействие с пользователем должно осуществляться посредством case-меню. Требуется реализовать возможность оценки погрешности полученного результата. Необходимо использовать процедуры и функции там, где это целесообразно.
	
	\section*{Решение}
	\begin{figure}[h]
		\centering
		\includegraphics[width=0.6\linewidth]{images/s-1}
	\end{figure}
	\begin{center}
		Рисунок 1 – Подпрограмма <<Кривая>>
	\end{center}
	
	\pagebreak
	\begin{figure}[h]
		\centering
		\includegraphics[width=0.6\linewidth]{images/s-2}
	\end{figure}
	\begin{center}
		Рисунок 2 – Подпрограмма <<Первообразная>>
	\end{center}
	
	\begin{figure}[h]
		\centering
		\includegraphics[width=0.6\linewidth]{images/s-4}
	\end{figure}
	\begin{center}
		Рисунок 3 – Подпрограмма <<Левый прямоугольник>>
	\end{center}
	
	\pagebreak
	\begin{figure}[h]
		\centering
		\includegraphics[width=0.6\linewidth]{images/s-3}
	\end{figure}
	\begin{center}
		Рисунок 4 – Подпрограмма <<Ньютон>>
	\end{center}
	
	\begin{figure}[h]
		\centering
		\includegraphics[width=1\linewidth]{images/s-5}
	\end{figure}
	\begin{center}
		Рисунок 5 – Схема алгоритма программы
	\end{center}
	
	\newpage
	\begin{Verbatim}[tabsize=2]
#include <stdio.h>
#include <stdlib.h>
#include <math.h>
#define X -1.71619

float curve(float x) {
	return 2 * pow(x, 3) - 2 * pow(x, 2) + 0 * x + 16;
}

float antiderivative(float x) {
	return 0.5 * pow(x, 4) - 2.0 / 3.0 * pow(x, 3) + 16 * x;
}

float calc_newton(float a, float b) {
	return antiderivative(b) - antiderivative(a);
}

float left_rect(float a, float b, float h) {
	float s = 0.0;
	
	for (float i = a; i < b; i = i + h) {
		s += curve(i + h) * h;
	}
	
	return s;
}

void print_menu() {
	printf("\033[0d\033[2J");
	printf("1. Ввод нижнего предела\n");
	printf("2. Ввод верхнего предела\n");
	printf("3. Ввод шага интегрирования\n");
	printf("4. Рассчет интеграла\n");
	printf("5. Рассчет погрешности\n");
}

void print_input(int *choice) {
	printf(">  ");
	scanf("%d", &*choice);
}

int main() {
	int choice, is_a = 0, is_b = 0, is_h = 0;
	float a, b, h;
	
	print_menu();
	print_input(&choice);
	
	while(1)
	{
		switch (choice)
		{
			case 1:
				print_menu();
				printf("Нижний предел: ");
				scanf("%f", &a);
				
				if (is_b) while (a >= b) {
					printf("Введите корректный нижний предел: ");
					scanf("%f", &a);
				}
				if (a < X) a = X;
				is_a = 1;
				print_input(&choice);
				break;
			
			case 2:
				print_menu();
				printf("Верхний предел: ");
				scanf("%f", &b);
				
				if (is_a) while (a >= b) {
					printf("Введите корректный верхний предел: ");
					scanf("%f", &b);
				}
				is_b = 1;
				print_input(&choice);
				break;
				
				case 3:
				print_menu();
				printf("Шаг интегрирования: ");
				scanf("%f", &h), is_h = 1;
				print_input(&choice);
				break;
			
			case 4:
				print_menu();
				if (is_a && is_b && is_h) {
					printf("Площадь: %.2f\n", left_rect(a, b, h));
				} else {
					printf("Не введены пределы или шаг интегрирования\n");
				}
				print_input(&choice);
				break;
			
			case 5:
				print_menu();
				if (is_a && is_b && is_h) {
					printf("Метод левых прямоугольников: %.2f\n", 
							left_rect(a, b, h));
					printf("Метод Ньютона-Лейбница:      
							%.2f\n", calc_newton(a, b));
					printf("Абсолютная погрешность:      %.2f\n", 
							fabs(left_rect(a, b, h) - calc_newton(a, b)));
					printf("Относительная погрешность:   %.2f%%\n",
							fabs((left_rect(a, b, h) - calc_newton(a, b)) /
							calc_newton(a, b) * 100));
				} else {
					printf("Не введены пределы или шаг интегрирования\n");
				}
				print_input(&choice);
				break;
			
			default:
				print_menu();
				print_input(&choice);
				break;
		}
	}
	return 0;
}
	\end{Verbatim}
	
	\section*{Вывод}
	В ходе выполнения лабораторной работы удалось освоить синтаксис построения подпрограмм и способы передачи данных в них. Также удалось организовать минимальный пользовательский интерфейс. В результате была реализована программа, которая вычисляет площадь фигуры, ограниченной, а также возможность оценки погрешности полученного результата. 
	
\end{document}