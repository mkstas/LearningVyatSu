\documentclass[a4paper,14pt]{extarticle}

\usepackage[a4paper,top=20mm,bottom=20mm,left=30mm,right=10mm]{geometry}
\usepackage[T1,T2A]{fontenc}
\usepackage[utf8]{inputenc}
\usepackage[russian]{babel}
\usepackage{indentfirst}
\usepackage{titlesec}
\usepackage{graphicx}
\usepackage{listings}

\renewcommand{\baselinestretch}{1.3}
\titleformat{\section}{\normalsize\bfseries}{\thesection}{1em}{}
\titleformat{\subsection}{\normalsize\bfseries}{\thesection}{1em}{}
\setlength{\parindent}{12.5mm}

\begin{document}

  \newpage\thispagestyle{empty}
  \begin{center}
    \MakeUppercase{
      Министерство науки и высшего образования Российской Федерации\\
      Федеральное государственное бюджетное образовательное учреждение высшего образования\\
      <<Вятский Государственный Университет>>\\
    }
    Институт математики и информационных систем\\
    Факультет автоматики и вычислительной техники\\
    Кафедра электронных вычислительных машин
  \end{center}
  \vfill

  \begin{center}
    Отчет по лабораторной работе №5\\
    по дисциплине\\
    <<Информатика>>\\
    <<Представление вещественных чисел в формате с плавающей точкой>>
  \end{center}
  \vfill

  \noindent
  \begin{tabular}{ll}
    Выполнил студент гр. ИВТб-1301-05-00 \hspace{5mm} &
    \rule[-1mm]{25mm}{0.10mm}\,/Макаров С.А./\\
    
    Руководитель доцент кафедры ЭВМ & \rule[-1mm]{25mm}{0.10mm}\,/Коржавина А.С./\\
  \end{tabular}

  \vfill
  \begin{center}
    Киров 2024
  \end{center}

  \newpage
  \section*{Цель}
  Цель лабораторной работы: закрепить на практике знания форматах представления числовой информации. Написать программы, решающие описанные ниже задачи.

  \section*{Задание}
  \begin{enumerate}
    \item Представить число в формате с плавающей точкой в n-разрядной сетке. Формат аналогичен IEEE 754. На входе: вещественное число в десятичной системе счисления, разрядность сетки, число разрядов мантиссы. На выходе: строка, отображающая введенное число в формате с плавающей точкой.
    
    \item Представить число в формате с плавающей точкой в n-разрядной сетке. Нормализация мантиссы дробная, формат с порядком, последовательность отображения – знак, мантисса, порядок. На входе: вещественное число в десятичной системе счисления, разрядность сетки, число разрядов мантиссы. На выходе: строка, отображающая введенное число в формате с плавающей точкой.
    
    \item Представить число в формате с плавающей точкой в n-разрядной сетке. Нормализация мантиссы дробная, формат с характеристикой, последовательность отображения – знак, мантисса, характеристика. На входе: вещественное число в десятичной системе счисления, разрядность сетки, число разрядов мантиссы. На выходе: строка, отображающая введенное число в формате с плавающей точкой. 
  \end{enumerate}

  \newpage
  \section*{Решение}
  Для решения представленных задач создадим подпрограмму побитового вывода десятичного числа, представленная на рисунке 1. Исходный код подпрограммы на языке C представлен в приложении А.

  \begin{figure}[h]
    \centering
    \includegraphics[width=0.65\linewidth]{schemes/s-p}
  \end{figure}
  \begin{center}
    Рисунок 1 – Схема алгоритма подпрограммы <<Побитовый вывод>>
  \end{center}

  \pagebreak
  \subsection*{Задание 1}
  Схема алгоритма для решения предлагаемой задачи представлена на рисунке 2. Исходный код на языке C представлен в приложении А.
  \begin{figure}[h]
    \centering
    \includegraphics[width=0.65\linewidth]{schemes/s-1}
  \end{figure}
  \begin{center}
    Рисунок 2 – Схема алгоритма задания 1
  \end{center}

  \pagebreak
  \subsection*{Задание 2}
  Схема алгоритма для решения предлагаемой задачи представлена на рисунке 3. Исходный код на языке C представлен в приложении А.
  \begin{figure}[h]
    \centering
    \includegraphics[width=0.65\linewidth]{schemes/s-2}
  \end{figure}
  \begin{center}
    Рисунок 3 – Схема алгоритма задания 2
  \end{center}

  \pagebreak
  \subsection*{Задание 3}
  Схема алгоритма для решения предлагаемой задачи представлена на рисунке 4. Исходный код на языке C представлен в приложении А.
  \begin{figure}[h]
    \centering
    \includegraphics[width=0.6\linewidth]{schemes/s-3}
  \end{figure}
  \begin{center}
    Рисунок 4 – Схема алгоритма задания 3
  \end{center}

  \section*{Вывод}
  В ходе выполнения лабораторной работы удалось закрепить на практике знания использования формата представления числовой информации. Были реализованы программы вывода числа в форматах IEEE-754, порядка, характеристики, написанных на языке C.

  \newpage
  \section*{Приложение А}
  \begin{lstlisting}[tabsize=2,basicstyle=\ttfamily]
#include <stdio.h>
#include <math.h>

void print_bin(int n, int b) {
  for (int i = b - 1; i >= 0; i--) {
    printf("%d", n >> i & 1);
  }
}

int main() {
  float x; int n, k;
  scanf("%f %d %d", &x, &n, &k);
  
  int* d = (int*) & x;
  int s = *d >> 31;
  int e = *d >> 23 & (int)pow(2, 8) - 1;
  int m = *d & (int)pow(2, 23) - 1;
  int p = e - (pow(2, 7) - 1);
  int c = p + 1 + pow(2, n - k - 2);
  e = p + (pow(2, n - k - 2) - 1);
  
  if (k < 23) {
    if (p > 0) {
      m = m / pow(2, k + p);
    } else {
      m = m / pow(2, 23 - k);
    }
  }
  \end{lstlisting}

  \pagebreak
  \section*{Продолжение приложения А}
  \begin{lstlisting}[tabsize=2,basicstyle=\ttfamily]
  print_bin(s, 1);
  print_bin(e, n - k - 1);
  print_bin(m, k);
  printf("\n");
  
  m = (m + pow(2, k)) / 2;
  
  print_bin(s, 1);
  print_bin(m, k);
  printf("%d", p + 1 < 0 ? 1 : 0);
  print_bin(abs(p + 1), n - k - 2);
  printf("\n");
  
  print_bin(s, 1);
  print_bin(m, k);
  print_bin(c, n - k - 1);
  printf("\n");
  
  return 0;
}
  \end{lstlisting}

\end{document}