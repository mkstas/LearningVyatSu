\documentclass[a4paper,14pt]{extarticle}

\usepackage[a4paper,top=20mm,bottom=20mm,left=30mm,right=10mm]{geometry}
\usepackage[T1,T2A]{fontenc}
\usepackage[utf8]{inputenc}
\usepackage[russian]{babel}
\usepackage{indentfirst}
\usepackage{titlesec}
\usepackage{graphicx}
\usepackage{listings}

\renewcommand{\baselinestretch}{1.3}
\titleformat{\section}{\normalsize\bfseries}{\thesection}{1em}{}
\titleformat{\subsection}{\normalsize\bfseries}{\thesection}{1em}{}
\setlength{\parindent}{12.5mm}

\begin{document}

  \newpage\thispagestyle{empty}
  \begin{center}
    \MakeUppercase{
      Министерство науки и высшего образования Российской Федерации\\
      Федеральное государственное бюджетное образовательное учреждение высшего образования\\
      <<Вятский Государственный Университет>>\\
    }
    Институт математики и информационных систем\\
    Факультет автоматики и вычислительной техники\\
    Кафедра электронных вычислительных машин
  \end{center}
  \vfill

  \begin{center}
    Отчет по лабораторной работе №4\\
    по дисциплине\\
    <<Информатика>>\\
    <<Форматы представления числовой информации.\\Представление целых чисел>>
  \end{center}
  \vfill

  \noindent
  \begin{tabular}{ll}
    Выполнил студент гр. ИВТб-1301-05-00 \hspace{5mm} &
    \rule[-1mm]{25mm}{0.10mm}\,/Макаров С.А./\\
    
    Руководитель доцент кафедры ЭВМ & \rule[-1mm]{25mm}{0.10mm}\,/Коржавина А.С./\\
  \end{tabular}

  \vfill
  \begin{center}
    Киров 2024
  \end{center}

  \newpage
  \section*{Цель}
  Цель лабораторной работы: закрепить на практике знания форматах представления числовой информации. Написать программы, решающие описанные ниже задачи. Программы должны работать без ошибок на любых наборах входных данных.

  \section*{Задание}
  \begin{enumerate}
    \item Представить беззнаковое (неотрицательное) число в n-разрядной сетке. На входе через пробел: целое число в десятичной системе счисления, разрядность сетки. На выходе: строка, отображающая введенное число в разрядной сетке.
    
    \item  Представить число в прямом коде в n-разрядной сетке. На входе через пробел: целое число в десятичной системе счисления, разрядность сетки. На выходе: строка, отображающая введенное число в прямом коде.
    
    \item Представить число в дополнительном коде в n-разрядной сетке. На входе: целое число в десятичной системе счисления, разрядность сетки. На выходе: строка, отображающая введенное число в дополнительном коде.
    
    \item Представить число в обратном коде в n-разрядной сетке. На входе: целое число в десятичной системе счисления, разрядность сетки. На выходе: строка, отображающая введенное число в обратном коде.
    
    \item Определить расстояние по Хеммингу двух дополнительных кодов. Расстояние по Хеммингу -- количество знакопозиций, в которых отличаются два кода, например, для кода Грея расстояние по Хеммингу между соседними кодами равно 1. На входе: два целых числа в десятичной системе счисления, разрядность сетки. На выходе: число -- расстояние по Хеммингу между ДК введенных чисел.
  \end{enumerate}

  \section*{Решение}
  \subsection*{Задание 1}
  Схема алгоритма для решения предлагаемой задачи представлена на рисунке 1.

  \begin{figure}[h]
    \centering
    \includegraphics[width=0.8\linewidth]{schemes/s-1}
  \end{figure}
  \begin{center}
    Рисунок 1 – Схема алгоритма задания 1
  \end{center}

  Решение задачи на языке C представлено ниже.

  \begin{lstlisting}[tabsize=2,basicstyle=\ttfamily]
#include <stdio.h>
#include <math.h>
int main() {
  int K, n;
  scanf("%d %d", &K, &n);
  if (K < 0 || K > pow(2, n) - 1) {
    printf("Error");
    return 0;
  }
  
  for (int i = n - 1; i >= 0; i--) {
    printf("%d", K >> i & 1);
  }
  return 0;
}
  \end{lstlisting}

  \subsection*{Задание 2}
  Схема алгоритма для решения предлагаемой задачи представлена на рисунке 2.

  \begin{figure}[h]
    \centering
    \includegraphics[width=0.8\linewidth]{schemes/s-2}
  \end{figure}
  \begin{center}
    Рисунок 2 – Схема алгоритма задания 2
  \end{center}
  \pagebreak

  Решение задачи на языке C представлено ниже.

  \begin{lstlisting}[tabsize=2,basicstyle=\ttfamily]
#include <stdio.h>
#include <math.h>
int main() {
  int K, n;
  scanf("%d %d", &K, &n);
  if (K > pow(2, n) / 2 - 1 
      || K < (pow(2, n) / 2 - 1) * -1) {
    printf("Error");
    return 0;
  }
  printf("%d", K < 0 ? 1 : 0);
  K = abs(K);
  for (int i = n - 2; i >= 0; i--) {
    printf("%d", K >> i & 1);
  }
  return 0;
}
  \end{lstlisting}

  \pagebreak
  \subsection*{Задание 3}
  Схема алгоритма для решения предлагаемой задачи представлена на рисунке 3.

  \begin{figure}[h]
    \centering
    \includegraphics[width=0.8\linewidth]{schemes/s-3}
  \end{figure}
  \begin{center}
    Рисунок 3 – Схема алгоритма задания 3
  \end{center}

  Решение задачи на языке C представлено ниже.

  \begin{lstlisting}[tabsize=2,basicstyle=\ttfamily]
#include <stdio.h>
#include <math.h>
int main() {
  int K, n;
  scanf("%d %d", &K, &n);
  if (K > pow(2, n) / 2 - 1 
      || K < pow(2, n) / 2 * -1) {
    printf("Error");
    return 0;
  }
  for (int i = n - 1; i >= 0; i--) {
    printf("%d", K >> i & 1);
  }
  return 0;
}
  \end{lstlisting}

  \subsection*{Задание 4}
  Схема алгоритма для решения предлагаемой задачи представлена на рисунке 4.

  \begin{figure}[h]
    \centering
    \includegraphics[width=0.8\linewidth]{schemes/s-4}
  \end{figure}
  \begin{center}
    Рисунок 4 – Схема алгоритма задания 4
  \end{center}
  \pagebreak

  Решение задачи на языке C представлено ниже.

  \begin{lstlisting}[tabsize=2,basicstyle=\ttfamily]
#include <stdio.h>
#include <math.h>
int main() {
  int K, n;
  scanf("%d %d", &K, &n);
  if (K > pow(2, n) / 2 - 1 
      || K < (pow(2, n) / 2 - 1) * -1) {
    printf("Error");
    return 0;
  }
  if (K < 0) {
    K = abs(K);
    K = ~K;
  }
  for (int i = n - 1; i >= 0; i--) {
    printf("%d", K >> i & 1);
  }
  return 0;
}
  \end{lstlisting}

  \pagebreak
  \subsection*{Задание 5}
  Схема алгоритма для решения предлагаемой задачи представлена на рисунке 5.

  \begin{figure}[h]
    \centering
    \includegraphics[width=0.6\linewidth]{schemes/s-5}
  \end{figure}
  \begin{center}
    Рисунок 5 – Схема алгоритма задания 5
  \end{center}

  Решение задачи на языке C представлено ниже.

  \begin{lstlisting}[tabsize=2,basicstyle=\ttfamily]
#include <stdio.h>
int main() {
  int K, M, n;
  scanf("%d %d %d", &K, &M, &n);
  int cnt = 0;
  for (int i = 0; i < n; i++) {
    cnt += K >> i & 1 ^ M >> i & 1;
  }
  printf("%d", cnt);
  return 0;
}
  \end{lstlisting}

  \section*{Вывод}
  В ходе выполнения лабораторной работы удалось закрепить на практике знания использования различных систем счисления, реализовав алгоритмы работы с целыми и вещественными числами в различных системах счисления.

\end{document}