\documentclass[a4paper,14pt]{extarticle}

\usepackage[a4paper,top=20mm,bottom=20mm,left=30mm,right=10mm]{geometry}
\usepackage[T1,T2A]{fontenc}
\usepackage[utf8]{inputenc}
\usepackage[russian]{babel}
\usepackage{indentfirst}
\usepackage{titlesec}
\usepackage{graphicx}
\usepackage{nicematrix}

\renewcommand{\baselinestretch}{1.3}
\titleformat{\section}{\normalsize\bfseries}{\thesection}{1em}{}
\titleformat{\subsection}{\normalsize\bfseries}{\thesection}{1em}{}
\setlength{\parindent}{12.5mm}

\begin{document}
  
  \newpage\thispagestyle{empty}
  \begin{center}
    \MakeUppercase{
      Министерство науки и высшего образования Российской Федерации\\
      Федеральное государственное бюджетное образовательное учреждение высшего образования\\
      <<Вятский Государственный Университет>>\\
    }
    Институт математики и информационных систем\\
    Факультет автоматики и вычислительной техники\\
    Кафедра электронных вычислительных машин
  \end{center}
  \vfill
  
  \begin{center}
    Отчет по лабораторной работе №7\\
    по дисциплине\\
    <<Информатика>>\\
    <<Построение комбинационных схем>>\\
    Вариант 10
  \end{center}
  \vfill
  
  \noindent
  \begin{tabular}{ll}
    Выполнил студент гр. ИВТб-1301-05-00 \hspace{5mm} &
    \rule[-1mm]{25mm}{0.10mm}\,/Макаров С.А./\\
    
    Руководитель доцент кафедры ЭВМ & \rule[-1mm]{25mm}{0.10mm}\,/Коржавина А.С./\\
  \end{tabular}
  
  \vfill
  \begin{center}
    Киров 2024
  \end{center}
  
  \newpage
  \section*{Цель}
  Цель лабораторной работы: закрепить на практике знания о минимизации системы булевых функций и получить навыки реализации простейших арифметических устройств.
  
  \section*{Задание}
  \begin{enumerate}
    \item Выполнить минимизацию булевых функций, представить функции различных базисах – основном логическом базисе (И, ИЛИ, НЕ) или в базисе Шеффера (И-НЕ) в соответствии с вариантом, после чего построить схему в системе Logisim и выполнить проверку.
    
    \item Построить четырехразрядный полный сумматор, складывающий 2 двоичных четырехразрядных числа и учитывающий единицу переноса. Построить схему сумматора в Logisim, проверить его работоспособность.
    
    \item Построить четырехразрядный умножитель, перемножающий 2 двоичных четырехразрядных числа. Построить схему умножителя в Logisim, проверить его работоспособность.  Допускается использование следующих логических элементов: И, ИЛИ, НЕ, И-НЕ, ИЛИ-НЕ,  сложение по модулю 2, эквивалентность.
    
    \item Построить 16-разрядный сумматор со схемами ускоренного переноса.  Построить схему сумматора в Logisim, проверить его работоспособность.  Допускается использование следующих логических элементов: И, ИЛИ, НЕ, И-НЕ, ИЛИ-НЕ,  сложение по модулю 2, эквивалентность.
  \end{enumerate}
  
  \newpage
  \section*{Решение}
  \subsection*{Задание 1}
  Таблица истинности для функции 1 представлена на таблице 1.
  
  \noindent Таблица 1 -- Таблица истинности функции 1 \\
  \begin{NiceTabular}{cccc}[hvlines]
    $x_1$ & $x_2$ & $x_3$ & $f$ \\
    0 & 0 & 0 & 0 \\
    0 & 0 & 1 & 0 \\
    0 & 1 & 0 & 0 \\
    0 & 1 & 1 & 1 \\
    1 & 0 & 0 & 1 \\
    1 & 0 & 1 & 0 \\
    1 & 1 & 0 & 0 \\
    1 & 1 & 1 & 1
  \end{NiceTabular}\\
  
  Диаграмма Вейча-Карно для минимизации функции 1 представлена на таблице 2.
  
  \noindent Таблица 2 -- Диаграмма Вейча-Карно функции 1 \\
  \begin{NiceTabular}{cccccc}[hvlines]
    & $x_2$ & 0 & 0 & 1 & 1 \\
    & $x_3$ & 0 & 1 & 1 & 0 \\
    $x_1$ & & & & & \\
    0 & & & & 1 & \\
    1 & & 1 & & 1 &
  \end{NiceTabular} \\
  
  Минимизированная функция: $f=x_1\overline{x}_2\overline{x}_3\lor x_2x_3$. Комбинационная схема представлена на рисунке 1.
  
  \begin{figure}[h]
    \centering
    \includegraphics[width=0.5\linewidth]{images/s-1-1}
  \end{figure}
  \begin{center}
    Рисунок 1 – Комбинационная схема функции 1
  \end{center}
  
  \pagebreak
  Таблица истинности для функции 2 представлена на таблице 3.
  
  \noindent Таблица 3 -- Таблица истинности функции 2 \\
  \begin{NiceTabular}{cccc}[hvlines]
    $x_1$ & $x_2$ & $x_3$ & $f$ \\
    0 & 0 & 0 & 0 \\
    0 & 0 & 1 & 0 \\
    0 & 1 & 0 & 1 \\
    0 & 1 & 1 & 1 \\
    1 & 0 & 0 & 1 \\
    1 & 0 & 1 & 1 \\
    1 & 1 & 0 & 0 \\
    1 & 1 & 1 & 0
  \end{NiceTabular}\\
  
  Диаграмма Вейча-Карно для минимизации функции 2 представлена на таблице 4.
  
  \noindent Таблица 4 -- Диаграмма Вейча-Карно функции 2 \\
  \begin{NiceTabular}{cccccc}[hvlines]
    & $x_2$ & 0 & 0 & 1 & 1 \\
    & $x_3$ & 0 & 1 & 1 & 0 \\
    $x_1$ & & & & & \\
    0 & & & & 1 & 1 \\
    1 & & 1 & 1 & &
  \end{NiceTabular}\\
  
  Минимизированная функция в базисе Шеффера: $f=x_1\overline{x}_2\lor\overline{x}_1x_2=\overline{
    \overline{x_1\overline{x_2x_2}}\;
    \overline{\overline{x_1x_1}x_2}
  }$. Комбинационная схема представлена на рисунке 2.
  
  \begin{figure}[h]
    \centering
    \includegraphics[width=0.5\linewidth]{images/s-1-2}
  \end{figure}
  \begin{center}
    Рисунок 2 – Комбинационная схема функции 2
  \end{center}
  
  \pagebreak
  Таблица истинности для функции 3 представлена на таблице 5.
  
  \noindent Таблица 5 -- Таблица истинности функции 3 \\
  \begin{NiceTabular}{ccccc}[hvlines]
    $x_1$ & $x_2$ & $x_3$ & $x_4$ & $f$ \\
    0 & 0 & 0 & 0 & 1 \\
    0 & 0 & 0 & 1 & 1 \\
    0 & 0 & 1 & 0 & 0 \\
    0 & 0 & 1 & 1 & 0 \\
    0 & 1 & 0 & 0 & 1 \\
    0 & 1 & 0 & 1 & 0 \\
    0 & 1 & 1 & 0 & 0 \\
    0 & 1 & 1 & 1 & 0 \\
    1 & 0 & 0 & 0 & 1 \\
    1 & 0 & 0 & 1 & 0 \\
    1 & 0 & 1 & 0 & 1 \\
    1 & 0 & 1 & 1 & 0 \\
    1 & 1 & 0 & 0 & 1 \\
    1 & 1 & 0 & 1 & 1 \\
    1 & 1 & 1 & 0 & 0 \\
    1 & 1 & 1 & 1 & 1
  \end{NiceTabular}\\
  
  Диаграмма Вейча-Карно для минимизации функции 3 представлена на таблице 6.
  
  \noindent Таблица 6 -- Диаграмма Вейча-Карно функции 3 \\
  \begin{NiceTabular}{ccccccc}[hvlines]
    & & $x_3$ & 0 & 0 & 1 & 1 \\
    & & $x_4$ & 0 & 1 & 1 & 0 \\
    $x_1$ & $x_2$ & & & & & \\
    0 & 0 & & 1 & 1 & & \\
    0 & 1 & & 1 & & & \\
    1 & 1 & & 1 & 1 & 1 & \\
    1 & 0 & & 1 & & & 1 \\
  \end{NiceTabular} \\
  
  Минимизированная функция: $f=x_1x_2x_4\lor x_1\overline{x}_2\overline{x}_4\lor \overline{x}_1\overline{x}_2\overline{x}_3\lor \overline{x}_3\overline{x}_4$. Комбинационная схема представлена на рисунке 3.
  
  \pagebreak
  \begin{figure}[h]
    \centering
    \includegraphics[width=0.5\linewidth]{images/s-1-3}
  \end{figure}
  \begin{center}
    Рисунок 3 – Комбинационная схема функции 3
  \end{center}
  
  Таблица истинности для функции 4 представлена на таблице 7.
  
  \noindent Таблица 7 -- Таблица истинности функции 4 \\
  \begin{NiceTabular}{ccccc}[hvlines]
    $x_1$ & $x_2$ & $x_3$ & $x_4$ & $f$ \\
    0 & 0 & 0 & 0 & 1 \\
    0 & 0 & 0 & 1 & 0 \\
    0 & 0 & 1 & 0 & 0 \\
    0 & 0 & 1 & 1 & 1 \\
    0 & 1 & 0 & 0 & 0 \\
    0 & 1 & 0 & 1 & 1 \\
    0 & 1 & 1 & 0 & 0 \\
    0 & 1 & 1 & 1 & 0 \\
    1 & 0 & 0 & 0 & 1 \\
    1 & 0 & 0 & 1 & 0 \\
    1 & 0 & 1 & 0 & 0 \\
    1 & 0 & 1 & 1 & 1 \\
    1 & 1 & 0 & 0 & 1 \\
    1 & 1 & 0 & 1 & 0 \\
    1 & 1 & 1 & 0 & 1 \\
    1 & 1 & 1 & 1 & 1
  \end{NiceTabular}\\
  
  \pagebreak
  Диаграмма Вейча-Карно для минимизации функции 4 представлена на таблице 8.
  
  \noindent Таблица 8 -- Диаграмма Вейча-Карно функции 4 \\
  \begin{NiceTabular}{ccccccc}[hvlines]
    & & $x_3$ & 0 & 0 & 1 & 1 \\
    & & $x_4$ & 0 & 1 & 1 & 0 \\
    $x_1$ & $x_2$ & & & & & \\
    0 & 0 & & 1 & & 1 & \\
    0 & 1 & & & 1 & & \\
    1 & 1 & & 1 & & 1 & 1 \\
    1 & 0 & & 1 & & 1 & \\
  \end{NiceTabular} \\
  
  Минимизированная функция в базисе Шеффера:\\
  $f=\overline{
    \overline{\overline{x_1x_1}x_2\;\overline{x_3x_3}x_4}\;
    \overline{\overline{x_2x_2}\;\overline{x_3x_3}\;\overline{x_4x_4}}\;
    \overline{x_1\overline{x_3x_3}\;\overline{x_4x_4}}\;
    \overline{x_1x_2x_3}\;
    \overline{\overline{x_2x_2}x_3x_4}\;
    \overline{x_1x_3x_4}
  }$. Комбинационная схема представлена на рисунке 2.
  
  \begin{figure}[h]
    \centering
    \includegraphics[width=0.5\linewidth]{images/s-1-4}
  \end{figure}
  \begin{center}
    Рисунок 4 – Комбинационная схема функции 4
  \end{center}
  
  \newpage
  \subsection*{Задание 2}
  Комбинационная схема четырехразрядного сумматора состоит из одноразрядных сумматоров, представленные на рисунке 5. Схема четырехразрядного сумматора представлена на рисунке 6.
  
  \begin{figure}[h]
    \centering
    \includegraphics[width=0.5\linewidth]{images/s-2-1}
  \end{figure}
  \begin{center}
    Рисунок 5 – Одноразрядный сумматор
  \end{center}
  
  \begin{figure}[h]
    \centering
    \includegraphics[width=1\linewidth]{images/s-2-2}
  \end{figure}
  \begin{center}
    Рисунок 6 – Четырехразрядный сумматор
  \end{center}
  
  \pagebreak
  \subsection*{Задание 3}
  Комбинационная схема четырехразрядного умножителя представлена на рисунке 7.
  \begin{figure}[h]
    \centering
    \includegraphics[width=0.9\linewidth]{images/s-3}
  \end{figure}
  \begin{center}
    Рисунок 7 – Четырехразрядный умножитель
  \end{center}
  
  \pagebreak
  \subsection*{Задание 4}
  Комбинационная схема сумматора для ускоренного переноса представлена на рисунке 8, схема ускоренного переноса на рисунке 9, 4-разрядный сумматор с ускоренным переносом на рисунке 10, 16-разрядный сумматор представлен на рисунке 11.
  
  \begin{figure}[h]
    \centering
    \includegraphics[width=0.35\linewidth]{images/s-4-1}
  \end{figure}
  \begin{center}
    Рисунок 8 – Сумматор для ускоренного переноса
  \end{center}
  
  \begin{figure}[h]
    \centering
    \includegraphics[width=0.6\linewidth]{images/s-4-2}
  \end{figure}
  \begin{center}
    Рисунок 9 – Схема ускоренного переноса
  \end{center}
  
  \pagebreak
  \begin{figure}[h]
    \centering
    \includegraphics[width=0.9\linewidth]{images/s-4-3}
  \end{figure}
  \begin{center}
    Рисунок 10 – 4-разрядный сумматор с ускоренным переносом
  \end{center}
  
  \begin{figure}[h]
    \centering
    \includegraphics[width=1\linewidth]{images/s-4-4}
  \end{figure}
  \begin{center}
    Рисунок 10 – 16-разрядный сумматор с ускоренным переносом
  \end{center}
  
  \newpage
  \section*{Вывод}
  В ходе лабораторной работы были минимизированы функции с помощью диаграммы Вейча-Корно, а также были построены их комбинационные схемы. Также были реализованы схемы 4-разрядного сумматора, 4-разрядного умножителя, 16-разрядного сумматора с ускоренным переносом.
  
\end{document}