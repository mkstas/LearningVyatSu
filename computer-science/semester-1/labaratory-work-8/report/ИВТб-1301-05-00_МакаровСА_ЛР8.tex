\documentclass[a4paper,14pt]{extarticle}

\usepackage[a4paper,top=20mm,bottom=20mm,left=30mm,right=10mm]{geometry}
\usepackage[T1,T2A]{fontenc}
\usepackage[utf8]{inputenc}
\usepackage[russian]{babel}
\usepackage{indentfirst}
\usepackage{titlesec}
\usepackage{graphicx}

\renewcommand{\baselinestretch}{1.3}
\titleformat{\section}{\normalsize\bfseries}{\thesection}{1em}{}
\titleformat{\subsection}{\normalsize\bfseries}{\thesection}{1em}{}
\setlength{\parindent}{12.5mm}

\begin{document}
  
  \newpage\thispagestyle{empty}
  \begin{center}
    \MakeUppercase{
      Министерство науки и высшего образования Российской Федерации\\
      Федеральное государственное бюджетное образовательное учреждение высшего образования\\
      <<Вятский Государственный Университет>>\\
    }
    Институт математики и информационных систем\\
    Факультет автоматики и вычислительной техники\\
    Кафедра электронных вычислительных машин
  \end{center}
  \vfill
  
  \begin{center}
    Отчет по лабораторной работе №8\\
    по дисциплине\\
    <<Информатика>>\\
    <<Разработка последовательных схем (счетчиков)>>
  \end{center}
  \vfill
  
  \noindent
  \begin{tabular}{ll}
    Выполнил студент гр. ИВТб-1301-05-00 \hspace{5mm} &
    \rule[-1mm]{25mm}{0.10mm}\,/Макаров С.А./\\
    
    Руководитель доцент кафедры ЭВМ & \rule[-1mm]{25mm}{0.10mm}\,/Коржавина А.С./\\
  \end{tabular}
  
  \vfill
  \begin{center}
    Киров 2024
  \end{center}
  
  \newpage
  \section*{Цель}
  Цель лабораторной работы: закрепить на практике знания об элементах памяти и последовательных устройствах и получить навыки их реализации.
  
  \section*{Задание}
  \begin{enumerate}
    \item Построить схемы прямого (на +1) и обратного (на -1) 4-разрядных двоичных счетчиков на счетных (T) триггерах. Построить схемы счетчиков в Logisim, проверить их работоспособность.
    
    \item Построить схему прямого счетчика по модулю 10, то есть считающего в прямом направлении от 0 до 9 на счетных (T) триггерах. Построить схему счетчика в Logisim, проверить его работоспособность.
    
    \item Построить схему прямого счетчика по произвольному модулю N, то есть считающего в прямом направлении от 0 до N-1 на счетных (T) триггерах. Построить схему счетчика в Logisim, проверить его работоспособность.
    
    \item Построить схему прямого счетчика по произвольному модулю N, то есть считающего в прямом направлении от 0 до N-1 на D триггерах. Построить схему счетчика в Logisim, проверить его работоспособность.
    
    \item Построить схему прямого счетчика на +3. Счетчик увеличивает значение на +3, то есть счет идет 0 3 6 9 12 15 0 и т.д. на T триггерах. Построить схему счетчика в Logisim, проверить его работоспособность.
    
    \item Построить схему прямого счетчика на +5. Счетчик увеличивает значение на +5, то есть счет идет 0 5 10 15 20 25 30 0 и т.д. на T триггерах. Построить схему счетчика в Logisim, проверить его работоспособность.
    
  \end{enumerate}
  
  \newpage
  \section*{Решение}
  \subsection*{Задание 1}
  Комбинационная схема прямого 4-разрядного счетчика представлена на рисунке 1, схема обратного счетчика представлена на рисунке 2.
  
  \begin{figure}[h]
    \centering
    \includegraphics[width=0.8\linewidth]{images/s-1-1}
  \end{figure}
  \begin{center}
    Рисунок 1 – Прямой 4-разрядный счетчик
  \end{center}
  
  \begin{figure}[h]
    \centering
    \includegraphics[width=0.8\linewidth]{images/s-1-2}
  \end{figure}
  \begin{center}
    Рисунок 2 – Обратный 4-разрядный счетчик
  \end{center}
  
  \subsection*{Задание 2}
  Комбинационная схема прямого счетчика по модулю 10 представлена на рисунке 3.
  
  \begin{figure}[h]
    \centering
    \includegraphics[width=0.8\linewidth]{images/s-2}
  \end{figure}
  \begin{center}
    Рисунок 3 – Прямой счетчик по модулю 10
  \end{center}
  
  \subsection*{Задание 3}
  Комбинационная схема сброса счетчика представлена на рисунке 4. Схема прямого счетчика по произвольному модулю представлена на рисунке 5.
  
  \begin{figure}[h]
    \centering
    \includegraphics[width=0.4\linewidth]{images/s-3-1}
  \end{figure}
  \begin{center}
    Рисунок 4 – Схема сброса счетчика
  \end{center}
  
  \begin{figure}[h]
    \centering
    \includegraphics[width=0.8\linewidth]{images/s-3-2}
  \end{figure}
  \begin{center}
    Рисунок 5 – Прямой счетчик по произвольному модулю
  \end{center}
  
  \pagebreak
  \subsection*{Задание 4}
  Комбинационная схема прямого счетчика по произвольному модулю на D-триггерах представлена на рисунке 6
  
  \begin{figure}[h]
    \centering
    \includegraphics[width=0.8\linewidth]{images/s-4}
  \end{figure}
  \begin{center}
    Рисунок 6 – Прямой счетчик по произвольному модулю на D-триггерах
  \end{center}
  
  \subsection*{Задание 5}
  Комбинационная схема прямого счетчика на +3, построенного на T-триггерах, представлена на рисунке 7.
  
  \begin{figure}[h]
    \centering
    \includegraphics[width=0.8\linewidth]{images/s-5}
  \end{figure}
  \begin{center}
    Рисунок 7 – Прямой счетчик на +3
  \end{center}
  
  \pagebreak
  \subsection*{Задание 6}
  Комбинационная схема прямого счетчика на +5, построенного на T-триггерах, представлена на рисунке 8.
  
  \begin{figure}[h]
    \centering
    \includegraphics[width=0.8\linewidth]{images/s-6}
  \end{figure}
  \begin{center}
    Рисунок 8 – Прямой счетчик на +5
  \end{center}
  
  \section*{Вывод}
  В ходе лабораторной работы были закреплены на практике знания об элементах памяти и последовательных устройств. Были реализованы схемы прямого (на +1) и обратного (на -1) счетчика на T-триггерах, схема прямого счетчика по модулю 10, схема прямого счетчика по произвольному модулю на T и D триггерах, схемы прямых счетчиков на +3 и +5.
  
\end{document}