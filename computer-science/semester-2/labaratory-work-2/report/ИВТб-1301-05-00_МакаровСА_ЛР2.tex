\documentclass[a4paper,14pt]{extarticle}

\usepackage[a4paper,top=20mm,bottom=20mm,left=30mm,right=10mm]{geometry}
\usepackage[T1,T2A]{fontenc}
\usepackage[utf8]{inputenc}
\usepackage[russian]{babel}
\usepackage{indentfirst}
\usepackage{titlesec}
\usepackage{fancyvrb}
\usepackage{minted}

\renewcommand{\baselinestretch}{1.3}

\titleformat{\section}{\normalsize\bfseries}{\thesection}{1em}{}
\titleformat{\subsection}{\normalsize\bfseries}{\thesubsection}{1em}{}

\setlength{\parindent}{12.5mm}

\begin{document}

  \newpage\thispagestyle{empty}
  \begin{center}
    \MakeUppercase{
      Министерство науки и высшего образования Российской Федерации\\
      Федеральное государственное бюджетное образовательное учреждение высшего образования\\
      <<Вятский Государственный Университет>>\\
    }
    Институт математики и информационных систем\\
    Факультет автоматики и вычислительной техники\\
    Кафедра электронных вычислительных машин
  \end{center}
  \vfill
  
  \begin{center}
    Отчет по лабораторной работе №2\\
    по дисциплине\\
    <<Информатика>>\\
    <<Пьезоэлемент, микросхемы>>\\
    Вариант 3
  \end{center}
  \vfill
  
  \noindent
  \begin{tabular}{ll}
    Выполнил студент гр. ИВТб-1301-05-00 \hspace{5mm} &
    \rule[-1mm]{25mm}{0.10mm}\,/Макаров С.А./\\
    
    Руководитель преподаватель & \rule[-1mm]{25mm}{0.10mm}\,/Шмакова Н.А./\\
  \end{tabular}
  
  \vfill
  \begin{center}
    Киров 2025
  \end{center}

  \newpage
  \section*{\hspace{12.5mm}Цель работы}
  Цель работы: ознакомление с элементами Arduino IDE, изучение
  основ работы со средой для программирования, а также сборка схем с
  пьезодинамиком, микросхемами.

  \section*{\hspace{12.5mm}Задание}

  \begin{enumerate}
    \item Использовать любые две мелодии для пьезодинамика из вложения. Добавить световую индикацию из 3 светодиодов. Добавить кнопку для переключения мелодий.
    \item Терменвокс. При падении освещенности звук должен увеличиваться, звуковой сигнал непрерывен. Необходимо использовать частоты первой октавы.
    \item Мерзкое пианино. Необходимо собрать клавитуру из 5 кнопок, на которой можно сыграть несколько нот второй октавы. Кнопкт должны подключаться через стягивающий резистор, входной сигнал обрабатвается конструкцией switch/case.
    \item Перетягивание каната. Добавить по 2 светодиода каждому игроку (всего 14). Создать игру, в которой необходиму бестрее соперника перетянуть канат на свою сторону. Необходимо использовать 14 светодиодов, кнопки подключаются через подтягивающий резистор, индикаия выиграша на светодиоде. Для перехода к следующему светодиоду необходимо 4 нажатия.
  \end{enumerate}

  \newpage
  \section*{\hspace{12.5mm}Решение}
  \subsection*{\hspace{12.5mm}Задание 1}
  Выполнена живая сборка. Исходный код программы:

  \begingroup
  \fontsize{14pt}{10pt}\selectfont
  \linespread{1}
  \begin{minted}{Arduino}
#define BUZZER_PIN 2
#define BUTTON_PIN 3
#define LED_PIN_1 5
#define LED_PIN_2 6
#define LED_PIN_3 7
int songOne[] = {
  440, 440, 440, 349, 523,
  440, 349, 523, 440, 0,
  659, 659, 659, 698, 535,
  415, 349, 523, 440, 0,
  880, 440, 440, 880, 830,
  784, 740, 698, 740, 0,
  455, 622, 587, 554, 523,
  466, 523, 0,
  349, 415, 349, 440, 523,
  440, 523, 659, 0,
  880, 440, 440, 880, 830,
  784, 740, 698, 740, 0,
  455, 622, 587, 554, 523,
  466, 523, 0,
  349, 415, 349, 523, 440,
  349, 523, 440, 0
};
int tempOne[] = {
  500, 500, 500, 350, 150,
  500, 350, 150, 650, 500,
  500, 500, 500, 350, 150,
  500, 350, 150, 650, 500,
  500, 300, 150, 500, 325,
  175, 125, 125, 250, 325,
  250, 500, 325, 175, 125,
  125, 250, 350,
  250, 500, 350, 125, 500,
  375, 125, 650, 500,
  500, 300, 150, 500, 325,
  175, 125, 125, 250, 325,
  250, 500, 325, 175, 125,
  125, 250, 350,
  250, 500, 375, 125, 500,
  375, 125, 600, 650
};
int songTwo[] = {
  2637, 2637, 0, 2637,
  0, 2093, 2637, 0,
  3136, 0, 0, 0,
  1568, 0, 0, 0,
  2093, 0, 0, 1568,
  0, 0, 1319, 0,
  0, 1760, 0, 1976,
  0, 1865, 1760, 0,
  1568, 2637, 3136,
  3520, 0, 2794, 3136,
  0, 2637, 0, 2093,
  2349, 1976, 0, 0,
  2093, 0, 0, 1568,
  0, 0, 1319, 0,
  0, 1760, 0, 1976,
  0, 1865, 1760, 0,
  1568, 2637, 3136,
  3520, 0, 2794, 3136,
  0, 2637, 0, 2093,
  2349, 1976, 0, 0
};
int tempTwo[] = {
  80, 80, 80, 80,
  80, 80, 80, 80,
  80, 80, 80, 80,
  80, 80, 80, 80,
  80, 80, 80, 80,
  80, 80, 80, 80,
  80, 80, 80, 80,
  80, 80, 80, 80,
  110, 110, 100,
  80, 80, 80, 80,
  80, 80, 80, 80,
  80, 80, 80, 80,
  80, 80, 80, 80,
  80, 80, 80, 80,
  80, 80, 80, 80,
  80, 80, 80, 80,
  110, 110, 100,
  80, 80, 80, 80,
  80, 80, 80, 80,
  80, 80, 80, 80,
};
volatile bool isFirstMusic = true;
volatile bool isChangedMusic = false;
int counter = 0;
void setup() {
  Serial.begin(9600);
  pinMode(BUZZER_PIN, OUTPUT);
  pinMode(BUTTON_PIN, INPUT_PULLUP);
  pinMode(LED_PIN_1, OUTPUT);
  pinMode(LED_PIN_2, OUTPUT);
  pinMode(LED_PIN_3, OUTPUT);
  attachInterrupt(1, changeMusic, FALLING);
}
void loop() {
  if (isChangedMusic) {
      isChangedMusic = false;
    isFirstMusic = !isFirstMusic; 
  }
  if (isFirstMusic) {
    musicOne();
  } else {
    musicTwo();
  }
}
void changeMusic() {
  isChangedMusic = true;
}
void beep(int note, int duration) {
  tone(BUZZER_PIN, note, duration);
  if (counter % 2 == 0) {
    digitalWrite(LED_PIN_1, HIGH);
    delay(duration);
    digitalWrite(LED_PIN_1, LOW);
  } else if (counter % 3 == 0) {
    digitalWrite(LED_PIN_2, HIGH);
    delay(duration);
    digitalWrite(LED_PIN_2, LOW);
  } else {
    digitalWrite(LED_PIN_3, HIGH);
    delay(duration);
    digitalWrite(LED_PIN_3, LOW);
  }
  noTone(BUZZER_PIN);
  delay(50);
  counter++;
}
void musicOne() {
  int size = sizeof(songOne) / sizeof(int);
  for (int i = 0; i < size; i++) {
    if (isChangedMusic) {
      return;
    }
    beep(songOne[i], tempOne[i]);
  }
}
void musicTwo() {
  int size = sizeof(songTwo) / sizeof(int);
  for (int i = 0; i < size; i++) {
    if (isChangedMusic) {
      return;
    }
    beep(songTwo[i], tempTwo[i]);
  }
}
  \end{minted}
  \endgroup

  \subsection*{\hspace{12.5mm}Задание 2}
  Выполнена живая сборка. Исходный код программы:

  \begingroup
  \fontsize{14pt}{10pt}\selectfont
  \linespread{1}
  \begin{minted}{Arduino}
#define BUZZER_PIN 3
#define LDR_PIN A0
int notes[] = {261, 277, 293, 311, 329, 349, 369, 392, 415, 440, 466, 493};
int size = sizeof(notes) / sizeof(int);
void setup() {
  pinMode(BUZZER_PIN, OUTPUT);
}
void loop() {
  int val, index;
  val = constrain(analogRead(LDR_PIN), 49, 150);
  index = map(val, 49, 150, 0, size);
  tone(BUZZER_PIN, notes[index]);
}
  \end{minted}
  \endgroup

  \subsection*{\hspace{12.5mm}Задание 3}
  Выполнена живая сборка. Исходный код программы:

  \begingroup
  \fontsize{14pt}{10pt}\selectfont
  \linespread{1}
  \begin{minted}{Arduino}
#define BUZZER_PIN 13
#define FIRST_KEY_PIN 7
#define KEY_COUNT 5
int index = 0;
void setup() {
  pinMode(BUZZER_PIN, OUTPUT);
}
void loop() {
  int keyPin = index + FIRST_KEY_PIN;
  boolean keyUp = digitalRead(keyPin);
  if (digitalRead(keyPin)) {
    switch (index) {
      case 0:
        tone(BUZZER_PIN, 523, 20);
        break;
      case 1:
        tone(BUZZER_PIN, 587, 20);
        break;
      case 2:
        tone(BUZZER_PIN, 659, 20);
        break;
      case 3:
        tone(BUZZER_PIN, 698, 20);
        break;
      case 4:
        tone(BUZZER_PIN, 784, 20);
        break;
    }
  }
  index = (index + 1) % 5;
}
  \end{minted}
  \endgroup

  \subsection*{\hspace{12.5mm}Задание 4}
  Выполнена живая сборка. Исходный код программы:

  \begingroup
  \fontsize{14pt}{10pt}\selectfont
  \linespread{1}
  \begin{minted}{Arduino}
int ledPins[14] = {4, 5, 6, 7, 8, 9, 10, 11, 12, 13, A0, A1, A2, A3};
int isFirstLed = true;
volatile int score = 0;
void setup() {
  for (int i = 0; i < 14; i++)
  	pinMode(ledPins[i], OUTPUT);
  pinMode(A4, OUTPUT);
  attachInterrupt(1, clickButtonOne, FALLING);
  attachInterrupt(0, clickButtonTwo, FALLING);
}
void clickButtonOne() {
  score++;
  if (abs(score) == 4 && isFirstLed) {
    score += 4;
    isFirstLed = false;
  }
}
void clickButtonTwo() {
  score--;
  if (abs(score) == 4 && isFirstLed) {
    score -= 4;
    isFirstLed = false;
  }
}
void loop() {
  do {
    if (abs(score) > 7) {
      for (int i = 0; i < 14; i++) {
        if (score > 0) {
          digitalWrite(ledPins[i], i >= 6 && i <= 6 + score / 4);
        } else {
          digitalWrite(ledPins[i], i <= 7 && i >= 7 + score / 4);
        }
      }
    } else {
      digitalWrite(ledPins[6], HIGH);
      digitalWrite(ledPins[7], HIGH);
    }
  } while (abs(score) < 28);
  if (score <= -28) digitalWrite(ledPins[0], HIGH);
  digitalWrite(A4, HIGH);
}
  \end{minted}
  \endgroup

  \section*{\hspace{12.5mm}Вывод}
  В ходе выполнения лабораторной работы изучены основы работы в Arduino IDE, а также собраны схемы с пьезоэлементом в соответствии с вариантом задания.

\end{document}