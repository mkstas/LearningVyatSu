\documentclass[a4paper,14pt]{extarticle}

\usepackage[a4paper,top=20mm,bottom=20mm,left=30mm,right=10mm]{geometry}
\usepackage[T1,T2A]{fontenc}
\usepackage[utf8]{inputenc}
\usepackage[russian]{babel}
\usepackage{indentfirst}
\usepackage{titlesec}
\usepackage{verbatim}
\usepackage{fancyvrb}

\renewcommand{\baselinestretch}{1.3}

\titleformat{\section}{\normalsize\bfseries}{\thesection}{1em}{}
\titleformat{\subsection}{\normalsize\bfseries}{\thesubsection}{1em}{}

\setlength{\parindent}{12.5mm}

\begin{document}

  \newpage\thispagestyle{empty}
  \begin{center}
    \MakeUppercase{
      Министерство науки и высшего образования Российской Федерации\\
      Федеральное государственное бюджетное образовательное учреждение высшего образования\\
      <<Вятский Государственный Университет>>\\
    }
    Институт математики и информационных систем\\
    Факультет автоматики и вычислительной техники\\
    Кафедра электронных вычислительных машин
  \end{center}
  \vfill
  
  \begin{center}
    Отчет по лабораторной работе №1\\
    по дисциплине\\
    <<Информатика>>\\
    <<Светодидные индикаторы>>\\
    Вариант 15
  \end{center}
  \vfill
  
  \noindent
  \begin{tabular}{ll}
    Выполнил студент гр. ИВТб-1301-05-00 \hspace{5mm} &
    \rule[-1mm]{25mm}{0.10mm}\,/Макаров С.А./\\
    
    Руководитель преподаватель & \rule[-1mm]{25mm}{0.10mm}\,/Шмакова Н.А./\\
  \end{tabular}
  
  \vfill
  \begin{center}
    Киров 2025
  \end{center}

  \newpage
  \section*{\hspace{12.5mm}Цель работы}
  Цель работы: ознакомление с элементами Arduino IDE, изучение основ
  работы со средой для программирования, а также сборка схем со
  светодиодными индикаторами.

  \section*{\hspace{12.5mm}Задание}

  \begin{enumerate}
    \item Бегущий огонек. Cделать движение светодиодов по варианту. Начальное состояние -- выключено. Движение происходит с центра к крайним, чётные/нечётные.
    \item Пульсар. Начальное состояние -- выключено. Необходимо использовать полевой транзистор MOSFET p-канальный. Изменение состояний -- уменьшение до середны, сброс,уменьшение до минимального.
    \item Ночной светильник. Начальное состояние -- включено. В качестве входного сигнала использовать термиистор. Использовать резистор с сопротивлением 100 кОм. Яркость уменьшается в зависимости от значения на датчике.
    \item Кнопочный переключатель. Включение происходит по двум кнопкам (левая, правая). Выключение происходит по нажатию на правую кнопку.
    \item RGB светодиод. Без фиксации нажатия кнопки. Срабатывание кнопки по отпусканию. Без смешивания цветов. Необходимо использовать светодиод с обшим анодом.
  \end{enumerate}

  \newpage
  \section*{\hspace{12.5mm}Решение}
  \subsection*{\hspace{12.5mm}Задание 1}
  Выполнена живая сборка. Исходный код программы:

  \noindent
  \begin{Verbatim}[tabsize=2,fontsize=\small]
#define FIRST_LED_PIN 2
#define LAST_LED_PIN 11
int pairs[6][2] = { {4, 4}, {2, 6}, {0, 8}, {1, 9}, {2, 7}, {3, 5} };
void setup() {
  for (int pin = FIRST_LED_PIN; pin <= LAST_LED_PIN; ++pin)
    pinMode(pin, OUTPUT);
}
void loop() {
  unsigned int ms = millis();
  int index = (ms / 1000) % 6;
  digitalWrite(FIRST_LED_PIN + pairs[index][0], HIGH);
  digitalWrite(FIRST_LED_PIN + pairs[index][1], HIGH);
  delay(1000);
  digitalWrite(FIRST_LED_PIN + pairs[index][0], LOW);
  digitalWrite(FIRST_LED_PIN + pairs[index][1], LOW);
}
  \end{Verbatim}

  \subsection*{\hspace{12.5mm}Задание 2}
  Выполнена живая сборка. Исходный код программы:

  \noindent
  \begin{Verbatim}[tabsize=2,fontsize=\small]
#define CONTROL_PIN 9
int brightness = 255;
int state = 0;
void setup() {
  pinMode(CONTROL_PIN, OUTPUT);
}
void loop() {
  if (state == 0) {
    brightness--;
    if (brightness == 128) state = 1;
  } 
  if (state == 1) {
    brightness = 255;
    state = 2;
  }
  if (state == 2) {
    brightness--;
    if (brightness == 0) state = 3;
  }
  if (state == 3) {
    brightness = 255;
    state = 0;
  }
  analogWrite(CONTROL_PIN, brightness);
  delay(10);
}
  \end{Verbatim}

  \subsection*{\hspace{12.5mm}Задание 3}
  Выполнена живая сборка. Исходный код программы:

  \noindent
  \begin{Verbatim}[tabsize=2,fontsize=\small]
#define LED_PIN 9
#define THERMISTOR_PIN A0
void setup() {
  pinMode(LED_PIN, OUTPUT);
}
void loop() {
  int temperature = constrain(analogRead(THERMISTOR_PIN), 939, 960);
  int brightness = map(temperature, 939, 960, 0, 255);
  delay(10);
}
  \end{Verbatim}

  \subsection*{\hspace{12.5mm}Задание 4}
  Выполнена живая сборка. Исходный код программы:

  \noindent
  \begin{Verbatim}[tabsize=2,fontsize=\small]
#define LEFT_BUTTON_PIN 2
#define RIGHT_BUTTON_PIN 3
#define LED_PIN 13
boolean leftButtonUp = false;
boolean ledEnabled = false;
void setup() {
  pinMode(LED_PIN, OUTPUT);
  pinMode(LEFT_BUTTON_PIN, INPUT_PULLUP);
  pinMode(RIGHT_BUTTON_PIN, INPUT_PULLUP);
}
void loop() {
  if (digitalRead(LEFT_BUTTON_PIN) == LOW && !ledEnabled) {
    leftButtonUp = true;
    delay(200);
  }
  if (digitalRead(RIGHT_BUTTON_PIN) == LOW && leftButtonUp && !ledEnabled) {
    digitalWrite(LED_PIN, HIGH);
    ledEnabled = true;
    delay(200);
  }
  if (digitalRead(RIGHT_BUTTON_PIN) == LOW && ledEnabled) {
    digitalWrite(LED_PIN, LOW);
    leftButtonUp = false;
    ledEnabled = false;
    delay(200);
  }
}
  \end{Verbatim}

  \subsection*{\hspace{12.5mm}Задание 5}
  Выполнена живая сборка. Исходный код программы:

  \noindent
  \begin{Verbatim}[tabsize=2,fontsize=\small]
int buttonPins[] = {2, 3, 4};
int ledPins[] = {9, 10, 11};
int index = 0;
void setup() {
  for (int i = 0; i < 3; i++) {
    pinMode(buttonPins[i], INPUT_PULLUP);
    pinMode(ledPins[i], OUTPUT);
    digitalWrite(ledPins[i], HIGH);
  }
}
void loop() {
  if (digitalRead(buttonPins[index]) == LOW) {
    if (digitalRead(buttonPins[index]) == HIGH) {
      digitalWrite(ledPins[index], LOW);
      delay(100);
      digitalWrite(ledPins[index], HIGH);
    }
  } else {
    digitalWrite(ledPins[index], HIGH);
    index = (index + 1) % 3;
  }
}
  \end{Verbatim}

  \section*{\hspace{12.5mm}Вывод}
  В ходе выполнения лабораторной работы изучены основы работы в Arduino IDE, а также собраны схемы со светодиодными индикаторами в соответствии с вариантом задания.

\end{document}