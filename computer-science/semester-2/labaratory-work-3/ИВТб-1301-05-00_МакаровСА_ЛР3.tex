\documentclass[a4paper,14pt]{extarticle}

\usepackage[a4paper,top=20mm,bottom=20mm,left=30mm,right=10mm]{geometry}
\usepackage[T1,T2A]{fontenc}
\usepackage[utf8]{inputenc}
\usepackage[russian]{babel}
\usepackage{indentfirst}
\usepackage{titlesec}
\usepackage{fancyvrb}
\usepackage{minted}

\renewcommand{\baselinestretch}{1.3}

\titleformat{\section}{\normalsize\bfseries}{\thesection}{1em}{}
\titleformat{\subsection}{\normalsize\bfseries}{\thesubsection}{1em}{}

\setlength{\parindent}{12.5mm}

\begin{document}

  \newpage\thispagestyle{empty}
  \begin{center}
    \MakeUppercase{
      Министерство науки и высшего образования Российской Федерации\\
      Федеральное государственное бюджетное образовательное учреждение высшего образования\\
      <<Вятский Государственный Университет>>\\
    }
    Институт математики и информационных систем\\
    Факультет автоматики и вычислительной техники\\
    Кафедра электронных вычислительных машин
  \end{center}
  \vfill

  \begin{center}
    Отчет по лабораторной работе №3\\
    по дисциплине\\
    <<Информатика>>\\
    <<Пьезоэлемент, микросхемы>>\\
    Вариант 3
  \end{center}
  \vfill

  \noindent
  \begin{tabular}{ll}
    Выполнил студент гр. ИВТб-1301-05-00 \hspace{5mm} & \rule[-1mm]{25mm}{0.10mm}\,/Макаров С.А./ \\
    Руководитель преподаватель & \rule[-1mm]{25mm}{0.10mm}\,/Шмакова Н.А./ \\
  \end{tabular}

  \vfill
  \begin{center}
    Киров 2025
  \end{center}

  \newpage
  \section*{\hspace{12.5mm}Цель работы}

  \section*{\hspace{12.5mm}Задание}

  \newpage
  \section*{\hspace{12.5mm}Решение}
  \subsection*{\hspace{12.5mm}Задание 1}
  Выполнена живая сборка. Исходный код программы:

  \begingroup
    \fontsize{14pt}{10pt}\selectfont
    \linespread{1}
    \begin{minted}{Arduino}
int numberSegments[10]={
  0b00111111, 0b00001010, 0b01011101, 0b01011110, 0b01101010,
  0b01110110, 0b01110111, 0b00011010, 0b01111111, 0b01111110,
};
int n = 0;
void setup() {
  for (int i = 1; i < 22; i++)
    pinMode(i, OUTPUT);
}
void loop() {
  int number = n * (n + 1) * (n + 2) / 6;
  if (number > 999) {
    n = 0;
    number = 0;
  }
  int maskOne = numberSegments[number % 10];
  int maskTwo = numberSegments[number / 10 % 10];
  int maskThree = numberSegments[number / 100];
  for (int i = 0; i < 7; i++) {
    digitalWrite(i + 1, bitRead(maskOne, i));
    digitalWrite(i + 8, bitRead(maskTwo, i));
    digitalWrite(i + 15, bitRead(maskThree, i));
  }
  n++;
  delay(1000);
}
    \end{minted}
  \endgroup

  \section*{\hspace{12.5mm}Вывод}

\end{document}