\documentclass[a4paper,14pt]{extarticle}

\usepackage[a4paper,top=20mm,bottom=20mm,left=30mm,right=10mm]{geometry}
\usepackage[T1,T2A]{fontenc}
\usepackage[utf8]{inputenc}
\usepackage[russian]{babel}
\usepackage{indentfirst}
\usepackage{titlesec}
\usepackage{fancyvrb}
\usepackage{minted}

\renewcommand{\baselinestretch}{1.3}

\titleformat{\section}{\normalsize\bfseries}{\thesection}{1em}{}
\titleformat{\subsection}{\normalsize\bfseries}{\thesubsection}{1em}{}

\setlength{\parindent}{12.5mm}

\begin{document}

  \newpage\thispagestyle{empty}
  \begin{center}
    \MakeUppercase{
      Министерство науки и высшего образования Российской Федерации\\
      Федеральное государственное бюджетное образовательное учреждение высшего образования\\
      <<Вятский Государственный Университет>>\\
    }
    Институт математики и информационных систем\\
    Факультет автоматики и вычислительной техники\\
    Кафедра электронных вычислительных машин
  \end{center}
  \vfill

  \begin{center}
    Отчет по лабораторной работе №15\\
    по дисциплине\\
    <<Информатика>>\\
    <<Светодионые сборки>>\\
    Вариант 3
  \end{center}
  \vfill

  \noindent
  \begin{tabular}{ll}
    Выполнил студент гр. ИВТб-1301-05-00 \hspace{5mm} & \rule[-1mm]{25mm}{0.10mm}\,/Макаров С.А./ \\
    Руководитель преподаватель & \rule[-1mm]{25mm}{0.10mm}\,/Шмакова Н.А./ \\
  \end{tabular}

  \vfill
  \begin{center}
    Киров 2025
  \end{center}

  \newpage
  \section*{\hspace{12.5mm}Цель работы}
  Цель работы: Ознакомление с элементами Arduino IDE, изучение основ работы со средой для программирования, а также сборка схем с семисегментными индикаторами и регистрами.

  \section*{\hspace{12.5mm}Задание}
  \begin{enumerate}
    \item Секундомер. Необходимо использовать семисегментный нидикатор с общим катодом для отображения последовательности тетраэдальных чисел (A000292).
    \item Секундомер на 4511. Необходимо использовать семисегментный нидикатор с общим анодом и дешифратор 4511 для отображения последовательности чисел Прота (A080075).
    \item Сборка на базе счетчика нажатий 595. Необходимо использовать семисегментный нидикатор и регистр 595 для отображения последовательности двоичных чисел с весом Хемминга 2 (A018900) с помощью счетчика нажатий.
    \item Секундомер с драйвером 4026. Необходимо использовать семисегментный нидикатор и драйвер 4026 для отображения последовательности чисел Хиггса (A009003) с помощью счетчика нажатий.
    \item Счетчик нажатий на 3ЛС338А. Необходимо использовать семисегментный нидикатор 3ЛС338А для отображения последовательности счастливых чисел (A000959) с помощью счетчика нажатий.
  \end{enumerate}

  \newpage
  \section*{\hspace{12.5mm}Решение}
  \subsection*{\hspace{12.5mm}Задание 1}
  Выполнена живая сборка. Исходный код программы:

  \begingroup
    \fontsize{14pt}{10pt}\selectfont
    \linespread{1}
    \begin{minted}{Arduino}
int numberSegments[10]={
  0b00111111, 0b00001010, 0b01011101, 0b01011110, 0b01101010,
  0b01110110, 0b01110111, 0b00011010, 0b01111111, 0b01111110,
};
int n = 0;
void setup() {
  for (int i = 1; i < 22; i++)
    pinMode(i, OUTPUT);
}
void loop() {
  int number = n * (n + 1) * (n + 2) / 6;
  if (number > 999) {
    n = 0;
    number = 0;
  }
  int maskOne = numberSegments[number % 10];
  int maskTwo = numberSegments[number / 10 % 10];
  int maskThree = numberSegments[number / 100];
  for (int i = 0; i < 7; i++) {
    digitalWrite(i + 1, bitRead(maskOne, i));
    digitalWrite(i + 8, bitRead(maskTwo, i));
    digitalWrite(i + 15, bitRead(maskThree, i));
  }
  n++;
  delay(1000);
}
    \end{minted}
  \endgroup

  \section*{\hspace{12.5mm}Решение}
  \subsection*{\hspace{12.5mm}Задание 2}
  Выполнена живая сборка. Исходный код программы:

  \begingroup
    \fontsize{14pt}{10pt}\selectfont
    \linespread{1}
    \begin{minted}{Arduino}
int number = 3;
void setup() {
  for (int i = 3; i < 15; i++)
    pinMode(i, OUTPUT);
}
bool isProtNumber(int number) {
  int m = number - 1;
  int n = 0;
  while ((m & 1) == 0) {
    m >>= 1;
    n++;
  }
  return (m < (1 << n));
}
void loop() {
  if (number > 999) number = 3;
  if (isProtNumber(number)) {
    for (int i = 0; i < 4; i++) {
      digitalWrite(i + 3, bitRead(number % 10, i));
      digitalWrite(i + 7, bitRead(number / 10 % 10, i));
      digitalWrite(i + 11, bitRead(number / 100, i));
    }
    delay(1000);
  }
  number++;
}
    \end{minted}
  \endgroup

  \section*{\hspace{12.5mm}Решение}
  \subsection*{\hspace{12.5mm}Задание 3}
  Выполнена живая сборка. Исходный код программы:

  \begingroup
    \fontsize{14pt}{10pt}\selectfont
    \linespread{1}
    \begin{minted}{Arduino}
#define DATA_PIN 5
#define LATCH_PIN 6
#define CLOCK_PIN 7
#define BUTTON_PIN 4
int number = 3;
bool buttonWasUp = true;
int segments[10]= {
  0b01111101, 0b00100100, 0b01111010, 0b01110110, 0b00100111,
  0b01010111, 0b01011111, 0b01100100, 0b01111111, 0b01110111
};
void setup() {
  pinMode(DATA_PIN, OUTPUT);
  pinMode(CLOCK_PIN, OUTPUT);
  pinMode(LATCH_PIN, OUTPUT);
  pinMode(BUTTON_PIN, INPUT_PULLUP);
}
bool isHemming(int n) {
  for (int i = 1; i < n; i *= 2) {
    for (int j = 1; j < n; j *= 2) {
      if (i != j && (i + j) == n) {
        return true;
      }
    }
  }
  return false;
}
void loop() {
  if (buttonWasUp && !digitalRead(BUTTON_PIN)) {
    delay(10);
    if (!digitalRead(BUTTON_PIN)) {
      number = (number + 1) % 1000;
      while (!isHemming(number)) {
        number = (number + 1) % 1000;
      }
    }
  }
  buttonWasUp = digitalRead(BUTTON_PIN);
  digitalWrite(LATCH_PIN, LOW);
  shiftOut(DATA_PIN, CLOCK_PIN, LSBFIRST, segments[number / 100]);
  shiftOut(DATA_PIN, CLOCK_PIN, LSBFIRST, segments[number / 10 % 10]);
  shiftOut(DATA_PIN, CLOCK_PIN, LSBFIRST, segments[number % 10]);
  digitalWrite(LATCH_PIN, HIGH);
}
    \end{minted}
  \endgroup

  \section*{\hspace{12.5mm}Решение}
  \subsection*{\hspace{12.5mm}Задание 4}
  Выполнена живая сборка. Исходный код программы:

  \begingroup
    \fontsize{14pt}{10pt}\selectfont
    \linespread{1}
    \begin{minted}{Arduino}
#define CLOCK_PIN 8
#define RESET_PIN 9
#define BUTTON_PIN 4
#define LIMIT 141
bool buttonWasUp = true;
int numbers[100] = { 0 };
int k = 0;
int sequenceLength = 0;
void generateHiggsSequence() {
  bool is_prime[LIMIT + 1];
  for (int i = 2; i <= LIMIT; i++) {
    is_prime[i] = true;
  }
  for (int p = 2; p * p <= LIMIT; p++) {
    if (is_prime[p]) {
      for (int i = p * p; i <= LIMIT; i += p) {
        is_prime[i] = false;
      }
    }
  }
  int index = 0;
  for (int n = 5; n <= LIMIT; n++) {
    for (int p = 2; p <= n; p++) {
      if (is_prime[p] && n % p == 0 && p % 4 == 1) {
        numbers[index++] = n;
        break;
      }
    }
  }
  sequenceLength = index;
}
void setup() {
  pinMode(CLOCK_PIN, OUTPUT);
  pinMode(RESET_PIN, OUTPUT);
  pinMode(BUTTON_PIN, INPUT_PULLUP);
  digitalWrite(RESET_PIN, HIGH);
  digitalWrite(RESET_PIN, LOW);
  generateHiggsSequence();
}
void loop() {
  if (buttonWasUp && !digitalRead(BUTTON_PIN)) {
    delay(50);
    if (!digitalRead(BUTTON_PIN)) {
      if (k < sequenceLength) {
        updateDisplay(numbers[k]);
        k++;
        if (k >= sequenceLength) {
          k = 0;
          digitalWrite(RESET_PIN, HIGH);
          digitalWrite(RESET_PIN, LOW);
        }
      }
    }
  }
  buttonWasUp = digitalRead(BUTTON_PIN);
}
void updateDisplay(int num) {
    digitalWrite(RESET_PIN, HIGH);
    digitalWrite(RESET_PIN, LOW);
    for (int i = 0; i < num; i++) {
      digitalWrite(CLOCK_PIN, HIGH);
      digitalWrite(CLOCK_PIN, LOW);
    }
}
    \end{minted}
  \endgroup

  \section*{\hspace{12.5mm}Решение}
  \subsection*{\hspace{12.5mm}Задание 5}
  Выполнена живая сборка. Исходный код программы:

  \begingroup
    \fontsize{14pt}{10pt}\selectfont
    \linespread{1}
    \begin{minted}{Arduino}
#define DATA_PIN 5
#define LATCH_PIN 6
#define CLOCK_PIN 7
#define BUTTON_PIN 4
int number = 1;
bool buttonWasUp = true;
int segments[10]= {
  0b01111101, 0b00100100, 0b01111010, 0b01110110, 0b00100111,
  0b01010111, 0b01011111, 0b01100100, 0b01111111, 0b01110111
};
void setup() {
  pinMode(DATA_PIN, OUTPUT);
  pinMode(CLOCK_PIN, OUTPUT);
  pinMode(LATCH_PIN, OUTPUT);
  pinMode(BUTTON_PIN, INPUT_PULLUP);
}
bool is_lucky(int number) {
  int numbers[number * 2 + 2];
  int count = 0;
  int index = 1;
  for (int i = 1; i <= number * 2 + 2; i += 2) {
    numbers[count++] = i;
  }
  while (index < count) {
    int step = numbers[index];
    int new_count = 0;
    for (int i = 0; i < count; i++) {
      if ((i + 1) % step != 0) {
        numbers[new_count++] = numbers[i];
      }
    }
    count = new_count;
    index++;
  }
  for (int i = 0; i < count; i++) {
    if (numbers[i] == number) {
      return true;
    }
  }
  return false;
}
void loop() {
  if (buttonWasUp && !digitalRead(BUTTON_PIN)) {
    delay(10);
    if (!digitalRead(BUTTON_PIN)) {
      number = (number + 1) % 1000;
      while (!is_lucky(number)) {
        number = (number + 1) % 1000;
      }
    }
  }
  buttonWasUp = digitalRead(BUTTON_PIN);
  digitalWrite(LATCH_PIN, LOW);
  shiftOut(DATA_PIN, CLOCK_PIN, LSBFIRST, ~segments[number % 10]);
  shiftOut(DATA_PIN, CLOCK_PIN, LSBFIRST, ~segments[number / 10 % 10]);
  shiftOut(DATA_PIN, CLOCK_PIN, LSBFIRST, ~segments[number / 100]);
  digitalWrite(LATCH_PIN, HIGH);
}
    \end{minted}
  \endgroup

  \section*{\hspace{12.5mm}Вывод}
  В ходе лабораторной работы были изучены основы работы в среде Arduino IDE, а также собраны схемы с семисегментными индикаторами (с общим анодом и катодом), регистром 595, дешифратором 4511, драйвером 4026.
 
\end{document}