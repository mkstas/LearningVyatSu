\documentclass[a4paper,14pt]{extarticle}

\usepackage[a4paper,top=20mm,bottom=20mm,left=30mm,right=10mm]{geometry}
\usepackage[T1,T2A]{fontenc}
\usepackage[utf8]{inputenc}
\usepackage[russian]{babel}
\usepackage{indentfirst}
\usepackage{titlesec}
\usepackage{graphicx}
\usepackage{verbatim}
\usepackage{fancyvrb}

\renewcommand{\baselinestretch}{1.3}
\titleformat{\section}{\normalsize\bfseries}{\thesection}{1em}{}
\titleformat{\subsection}{\normalsize\bfseries}{\thesection}{1em}{}
\setlength{\parindent}{12.5mm}

\begin{document}

	\newpage\thispagestyle{empty}
	\begin{center}
		\MakeUppercase{
			Министерство науки и высшего образования Российской Федерации\\
			Федеральное государственное бюджетное образовательное учреждение высшего образования\\
			<<Вятский Государственный Университет>>\\
		}
		Институт математики и информационных систем\\
		Факультет автоматики и вычислительной техники\\
		Кафедра электронных вычислительных машин
	\end{center}
	\vfill
	
	\begin{center}
		Отчет по лабораторной работе №6\\
		по дисциплине\\
		<<Программирование>>\\
	\end{center}
	\vfill
	
	\noindent
	\begin{tabular}{ll}
		Выполнил студент гр. ИВТб-1301-05-00 \hspace{5mm} &
		\rule[-1mm]{25mm}{0.10mm}\,/Макаров С.А./\\
		
		Руководитель зав. кафедры ЭВМ & \rule[-1mm]{25mm}{0.10mm}\,/Долженкова М.Л./\\
	\end{tabular}
	
	\vfill
	\begin{center}
		Киров 2025
	\end{center}
	
	\newpage
	\section*{Цель}
	Цель лабораторной работы: изучение структуры и принципов организации программных модулей, закрепление навыков работы с динамической памятью, получение базовых навыков организации работы в режиме командной строки.
	
	\section*{Задание}
	\begin{enumerate}
		\item Написать программу для работы со структурой данных <<Очередь>>.
		\item Структура данных должна быть реализована на основе динамической памяти.
		\item Структура данных (поля и методы) должна быть описана в отдельном модуле.
		\item Работа со структурой должна осуществляться в режиме командной строки (с реализацией автодополнения и истории команд). Предусмотреть наглядную визуализацию содержимого структуры.
	\end{enumerate}
	
	\newpage
	\section*{Решение}
	
	\section*{Вывод}

\end{document}