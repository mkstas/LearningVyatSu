\documentclass[a4paper,14pt]{extarticle}

\usepackage[a4paper,top=20mm,bottom=20mm,left=30mm,right=10mm]{geometry}
\usepackage[T1,T2A]{fontenc}
\usepackage[utf8]{inputenc}
\usepackage[russian]{babel}
\usepackage{indentfirst}
\usepackage{titlesec}
\usepackage{graphicx}
\usepackage{verbatim}
\usepackage{fancyvrb}

\renewcommand{\baselinestretch}{1.3}
\titleformat{\section}{\normalsize\bfseries}{\thesection}{1em}{}
\titleformat{\subsection}{\normalsize\bfseries}{\thesection}{1em}{}
\setlength{\parindent}{12.5mm}

\begin{document}

  \newpage\thispagestyle{empty}
  \begin{center}
    \MakeUppercase{
      Министерство науки и высшего образования Российской Федерации\\
      Федеральное государственное бюджетное образовательное учреждение высшего образования\\
      <<Вятский Государственный Университет>>\\
    }
    Институт математики и информационных систем\\
    Факультет автоматики и вычислительной техники\\
    Кафедра электронных вычислительных машин
  \end{center}
  \vfill

  \begin{center}
    Отчет по лабораторной работе №4\\
    по дисциплине\\
    <<Управление данными>>\\
  \end{center}
  \vfill

  \noindent
  \begin{tabular}{ll}
    Выполнил студент гр. ИВТб-2301-05-00 \hspace{5mm} &
    \rule[-1mm]{25mm}{0.10mm}\,/Макаров С.А./\\
    
    Преподаватель & \rule[-1mm]{25mm}{0.10mm}\,/Клюкин В.Л./\\
  \end{tabular}

  \vfill
  \begin{center}
    Киров 2025
  \end{center}

  \newpage
  \section*{Цель}
  Цель лабораторной работы: познакомиться c библиотекой в C для связывания приложения с БД, изучить некоторые шаблоны проектирования, связанные с работой с БД, освоить на практике основы взаимодействия с БД под управлением PostgreSQL в приложении на C.

  \section*{Задание}
  При выполнении работы нужно использовать БД, созданную в предыдущих лабораторных работах. Необходимо создать приложение на языке программирования С с графическим интерфейсом.

  Требования к интерфейсу:
  \begin{itemize}
    \item[--] Названия колонок, кнопок, объектов ввода/вывода на русском языке (Например, ‘Имя’, а не ‘Name’),
    \item[--] Запретить ввод отрицательных значений (Например, цена не может быть отрицательной),
    \item[--] Ввод данных для выборки должен быть регистронезависимый (используйте функции UPPER или LOWER),
    \item[--] Для ввода даты, по возможности, использовать календарь.
  \end{itemize}

  Для любой одной таблицы, которая содержит внешний ключ на другую таблицу, приложение должно выполнять следующие функции:
  \begin{itemize}
    \item[--] Выводить, удалять и изменять данные таблицы,
    \item[--] В случае ввода уже имеющихся данных выводить сообщение об этом пользователю без записи данных в таблицу,
    \item[--] Удалять при подтверждении (Например, ‘Вы действительно уверены?’ Да/Нет),
    \item[--] Выполнять фильтр (выборку) по значениям строк. (Например, «Дата с … по …» или «Имя содержит …»).
  \end{itemize}

  Требования к реализации:
  \begin{itemize}
    \item[--] При добавлении новой строки внешний ключ выбирается из списка значений родительской таблицы (например, если таблица «Чек» ссылается на таблицу «Товар», нужно вывести список не id товара, а его название),
    \item[--] Сохранение или удаление строки должно быть реализовано с помощью функции PL/pgSQL
    \item[--] Фильтрация значений при поиске должна производиться через запрос, а не в полученной коллекции
    \item[--] Разрешается использование любого фреймворка,
    \item[--] При разработке можно использовать шаблоны проектирования, связанные с работой с БД
  \end{itemize}

  \pagebreak
  \section*{Решение}

  \section*{Вывод}

\end{document}